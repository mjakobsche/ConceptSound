\documentclass[12pt]{article}

\usepackage[polish]{babel}
\usepackage{csquotes}
\usepackage[a4paper,top=2.5cm,bottom=2.5cm,left=3cm,right=2cm]{geometry}
\usepackage{indentfirst}
\usepackage{lipsum}
\usepackage{amsmath}
\usepackage{graphicx}
\usepackage[colorlinks=true, allcolors=blue]{hyperref}
\usepackage{float}
\usepackage{fontspec}
\setmainfont{Times New Roman}
\setlength{\parindent}{1.25cm}
\linespread{1.5}
\setlength{\parskip}{6pt}
\usepackage[sorting=none]{biblatex}
\addbibresource{bibliography.bib}
\hypersetup{linkcolor=black}
\usepackage[all]{hypcap}
\addto\captionspolish{\renewcommand{\figurename}{Rys.}}
\addto\captionspolish{\renewcommand{\tablename}{Tab.}}
\renewcommand{\tableautorefname}{Tab.}
\renewcommand{\figureautorefname}{Rys.}
\renewcommand{\listfigurename}{Spis rysunków}
\renewcommand{\listtablename}{Spis tabel}
\newcommand{\image}[3][16cm]{
    \begin{figure}[H]
        \centering
        \includegraphics[width=#1]{media/#2}
        \caption{#3}
        \label{fig:2}
    \end{figure}
}

\begin{document}

\newpage
\clearpage
\tableofcontents

\newpage
\addcontentsline{toc}{section}{Wstęp}
\section*{Wstęp}

\newpage
\addcontentsline{toc}{section}{Cel pracy}
\section*{Cel pracy}
Celem pracy jest stworzenie aplikacji mobilnej, która wesprze początkowe fazy procesu aranżacji muzycznej,
stanowiąc podręczne narzędzie do wygodnego notowania koncepcji, pomysłów oraz wskazówek artykulacyjnych utworów muzycznych.
Jej zadaniem jest cyfrowa emulacja \enquote{zeszytu do nut} jako rodzaju muzycznego notesu, rozszerzając jego możliwości
dzięki wykorzystanej technologii — nie ograniczając się jedynie do zapisu nutowego oraz tekstowego,
lecz oddając w ręce użytkownika również narzędzia multimedialne, umożliwiające pracę z materiałem muzycznym.
Stąd proponowana jest nazwa aplikacji \enquote{Concept Sound}.
Unikatowość projektu opiera się o ukierunkowanie funkcjonalności aplikacji na procesy pomijane przez współczesne
oprogramowanie wspierające pracę muzyków.

\newpage
\addcontentsline{toc}{section}{Zakres pracy}
\section{Zakres pracy}
\subsection{Studium literaturowe}
\subsubsection{Rys historyczny dziedzin zagadnienia}
Projektowana aplikacja — pod kątem zasady działa — stanowi swoisty notatnik; pod kątem dziedziny jest on ukierunkowany na wspomaganie
muzycznego procesu twórczego.
Stąd stosowne jest przedstawienie rysu historycznego, nawiązując do obu tych aspektów oraz ich manifestacji na płaszczyźnie wirtualnej.

\textbf{Techniki kompozytorskie}
Jako pierwszy, należałoby przytoczyć zarys historii muzyki klasycznej, z naciskiem na charakter krajobrazu muzycznego
poszczególnych epok i tendencji popularnych wśród twórców w Europie. Ukazuje on znaczne zróżnicowanie w podejściu do procesu tworzenia.
\begin{itemize}
	\item \textbf{Średniowiecze}:
	      
	      Proces twórczy kompozytorów średniowiecza był głęboko osadzony w liturgicznym kontekście religijnym oraz w tradycji
	      chorału gregoriańskiego.
	      Kompozytorzy tego okresu opierali się na zawartych w nim melodiach, wykorzystując techniki kompozytorskie, takie jak organum,
	      melizmaty oraz schematy rytmiczne, aby urozmaicić chorał gregoriański.
	      Organum polegało na dodawaniu nowych dźwięków do melodii chorału, co pozwalało utworzyć wielowarstwowe melodie.
	      Melizmaty stosowane były jako melodyczne ozdobniki. Kluczową rolę odgrywał tekst utworów.
	\item \textbf{Renesans}:
	      
	      Punktem przełomowym dla epoki renesansu była polifonia — czyli tworzenie wielu równoległych głosów o różnych liniach melodycznych.
	      Szczególną popularnością cieszyła się wówczas technika imitacji — naśladownictwa.
	      Zaczęto stosować także bardziej skomplikowane rytmy, np. opierające się na trójdzielności oraz nowe skale muzyczne.
	\item \textbf{Barok}:
	      
	      Twórczość kompozytorów epoki baroku charakteryzowała się rozbudowaną ornamentacją.
	      Zaczęto stosować tzw. basso continuo, jako metodę prowadzenia akompaniamentu, oraz technikę kontrapunktu,
	      w której kontrastujące melodie tworzyły złożone struktury dźwiękowe.
	      Kompozytorzy baroku eksperymentowali z harmonią i dynamiką,
	\item \textbf{Klasycyzm}:
	      
	      W okresie klasycyzmu proces twórczy kompozytorów cechowała kontrastująca z barokiem precyzja i klarowność.
	      Zaczęto wręcz masowo stosować ścisłe formy, wśród których królowała forma sonatowa, określająca strukturę, tempo,
	      charakter i układ poszczególnych części utworów.
	      Wykorzystywane powtarzalne motywy oraz dążono do równowagi między poszczególnymi elementami składowymi utworów.
	\item \textbf{Romantyzm}:
	      
	      Znów w widocznym kontraście epoka romantyzmu przynosi duże oswobodzenie wyrazowe muzyki.
	      Kluczowym elementem motywującym twórczość staje się introspekcja, stąd utwory tego okresu nasycone są indywidualizmem
	      i wyrazistością emocjonalną.
	      Wciąż stosowane są pewne wzorce — gatunki, jak sonata czy fantazja, ale nie są one już tak ściśle określone,
	      stąd pojawiają się zróżnicowane eksperymenty kompozytorskie z nietypową harmonią, dynamicznymi kontrastami i rozbudowanymi
	      frazami melodycznymi.
	\item \textbf{XX wiek}:
	      
	      W XX wieku obserwowane jest szczególne bogactwo i różnorodność podejść do kompozycji.
	      Pojawiają się nurty impresjonizmu, ekspresjonizmu, neoklasycyzmu.
	      Muzyka staje się produktem licznych koncepcji, metod tworzenia — aleatoryzmu, opierającego się o losowość,
	      serializmu, kierowanego przetworzeniami matematycznymi; a także idei — tu można wymienić minimalizm.
	      Epoka ta przynosi jeszcze jeden przełom — komputery i ich zastosowanie w muzyce, jako narzędzia twórcze, nośniki i instrumenty.
\end{itemize}

\textbf{Notacja muzyczna}
Na przestrzeni epok kluczowym nośnikiem pozwalającym na utrwalenie melodii była notacja — być może najcenniejsze narzędzie kompozytorów.
Ona również ewoluowała i zmieniała się, dostosowując się do zaawansowania rozwoju muzyki.
Przyglądając się tej ewolucji, zauważalne jest bogactwo koncepcji zapisu melodii oraz jego zróżnicowane ukierunkowanie.
\begin{itemize}
	\item \textbf{Starożytność}:
	      
	      Najstarsza forma notacji muzycznej przypisywana jest Babilończykom (1400 rok p.n.e.).
	      Opiera się o zapis klinowy.
	      Podobnie późniejsze odkrycie — pochodzące ze Starożytnej Grecji — również opiera się o alfabet,
	      a także symbole określające długość dźwięku.
	\item \textbf{Średniowiecze}:
	      
	      Pojawia się wówczas notacja dedykowana stricte muzyce — neumy. Pierwsza notacja tego typu wywodzi się z Imperium Bizantyjskiego,
	      jednak aktualnie pod tym pojęciem rozumie się przede wszystkim inną, europejską notację wykorzystywaną początkowo w chorale
	      gregoriańskim — do określania wysokości dźwięków oraz ich czasu trwania.
	      Również wtedy pojawiła się solmizacja.
	\item \textbf{Renesans}:
	      
	      Epoka renesansu przynosi podwaliny współczesnej, ogólnej notacji nutowej, zaczynają być stosowane nuty oraz symbole
	      — jak na przykład klucze.
	      Na przestrzeni kolejnych epok zapis jest rozwijany, umożliwiając bardzo dokładne określanie zamierzonego przez twórcę przebiegu
	      melodycznego.
	      Jednocześnie pojawiają się tabulatury — zapis ukierunkowany pod konkretny instrument, dostosowany pod jego charakterystykę —
	      przykładowo organy, lutnie itd.
	\item \textbf{Barok}:
	      
	      Zaczyna być stosowany bas cyfrowany.
	      Jest to notacja numeryczna określające harmonię akompaniamentu utworów.
	      Rola ta przypadała zwykle konkretnym instrumentom — jak na przykład klawesynowi.
	      Zasada działania tego rodzaju notacji opierała się o akordy i relacje interwałowe pomiędzy poszczególnymi dźwiękami.
	\item \textbf{Klasycyzm i romantyzm}:
	      
	      Dalszy rozwój standaryzowanego zapisu nutowego, celem oznaczenia konkretnych wskazówek dla wykonawców utworów.
	\item \textbf{XX wiek}:
	      
	      Muzycy zaczynają eksperymentować z zapisem muzycznym, dostosowując go zupełnie do swoich potrzeb, tworząc warstwy abstrakcji,
	      własne "paradygmaty", odwzorowując w ten sposób skalę eksperymentalizmu tworzonej muzyki.
	\item \textbf{Współczesność}:
	      
	      Współcześnie korzystamy z wielu form zapisu muzycznego. Kluczową rolę odgrywa zapis nutowy,
	      lecz niektórzy artyści wciąż czasami sięgają po własne formy notacji, podobnie jak miało to miejsce w XX wieku.
	      Wielką popularnością cieszą się również tabulatury oraz wizualizacje komputerowe, z uwagi ta ich przystępność oraz zapis akordowy.
	      Jednocześnie w pewnych okolicznościach wciąż wykorzystywane są starsze metody zapisu — sięgające nawet do neum.
\end{itemize}

\textbf{Technologia współczesna}
W ramach tej sekcji, przedstawione zostanie oprogramowanie, w kolejności chronologicznej, które przynosiły przełomowe rozwiązania, doprowadzając do aktualnego stanu dziedziny.

\textbf{Wykorzystanie komputerów w tworzeniu muzyki}
\begin{itemize}
	\item \textbf{1951 - komputerowe generowanie dźwięku}:
	      
	      W 1951 roku inżynier komputerowy Max Mathews stworzył pierwszy program komputerowy do generowania dźwięku.
	      Program ten działał na komputerze IBM 704 i otworzył drzwi do eksperymentów z syntezą dźwięku przy użyciu komputerów.
	\item \textbf{1967 - kompozycja z wykorzystaniem komputera}:
	      
	      Program komputerowy "MUSIC I", tego samego twórcy, umożliwił kompozytorom tworzenie muzyki za pomocą kodu komputerowego.
	      Można określić ten moment jako początek muzyki komputerowej.
	\item \textbf{1980 - MIDI (Musical Instrument Digital Interface) i edycja dźwięku}:
	      
	      Wprowadzenie standardu MIDI umożliwiło komunikację między różnymi instrumentami muzycznymi a komputerami.
	      To znacząco ułatwiło produkcję i edycję muzyki przy użyciu oprogramowania.
	      W latach 80' pojawiły się też programy takie jak SoundEdit i Sound Forge,
	      które umożliwiły użytkownikom edycję dźwięku przy użyciu komputera.
	      Zyskały one popularność w dziedzinie produkcji muzycznej.
	\item \textbf{1990 - DAW (Digital Audio Workstation), syntezatory i samplowanie}:
	      
	      Powstanie programów DAW, takich jak Pro Tools, Cubase, Logic Pro, Ableton Live i FL Studio, umożliwiło profesjonalistom tworzenie,
	      edycję i produkcję muzyki na komputerach osobistych.
	      DAW stały się kluczowym narzędziem w przemyśle muzycznym.
	      Pojawiło się również oprogramowanie, które umożliwia emulację syntezatorów analogowych oraz samplowanie dźwięków.
	      Programy takie jak Native Instruments Kontakt i Serum stały się ważnymi narzędziami dla producentów muzycznych.
	\item \textbf{Lata 2000 - Rozwój wirtualnych instrumentów i efektów};
	      
	      Oprogramowanie takie jak Kontakt, Omnisphere czy Massive pozwoliło na tworzenie wirtualnych instrumentów i efektów dźwiękowych,
	      co znacząco rozszerzyło możliwości produkcji muzycznej.
	      Wciąż pojawiają się rozwiązania oferujące lepszą jakość i realizm dźwięku.
	\item \textbf{Współczesność}:
	      
	      Oferowana jest szeroka gama oprogramowania dla potrzeb muzyków, odpowiadająca na większość potrzeb zarówno profesjonalistów,
	      jak i amatorów.
	      Można tu wymienić programy ukierunkowane na produkcję muzyczną, tworzenie zapisów nutowych,
	      jak również narzędzia umożliwiające generowanie muzyki.
	      Wyraźny jest jednak brak rozwiązań ukierunkowanych na wstępne fazy tworzenia utworów oraz narzędzi umożliwiających szybkie
	      odnotowywanie pomysłów i koncepcji.
\end{itemize}
\textbf{Aplikacja jako narzędzie do notowania}
\begin{itemize}
	\item \textbf{1987 - Lotus Agenda}:
	      
	      Lotus Agenda, stworzony przez firmę Lotus Development Corporation oparty, był jednym z pierwszych programów do zarządzania notatkami.
	      Umożliwiał użytkownikom tworzenie notatek, organizowanie ich w kategorie i przeszukiwanie.
	\item \textbf{2000 - EverNote}:
	      
	      Evernote, założone przez Stepana Pachikova, stało się jednym z najbardziej rozpoznawalnych narzędzi do tworzenia notatek.
	      To narzędzie pozwala na tworzenie notatek w różnych formatach, w tym tekst, obrazy, dźwięki i wiele innych,
	      oraz synchronizację ich między różnymi urządzeniami.
	\item \textbf{2002 - Tomboy}:
	      
	      W miarę wzrostu popularności komputerów osobistych pojawiają się projekty open source, takie jak Tomboy,
	      które umożliwiające użytkownikom tworzenie notatek i notatników.
	\item \textbf{2003 - Microsoft OneNote}:
	      
	      Microsoft OneNote został wprowadzony jako część pakietu Microsoft Office w 2003 roku, ale jego historia sięga wcześniejszych lat.
	      Pierwotnie nazywany "OneNote 2003", ten program zyskał popularność jako narzędzie do tworzenia notatek na platformach Windows.
	      W kolejnych edycjach umożliwiał użytkownikom innowacyjne organizowanie notatek w ramach "tablicy".
	\item \textbf{2008 - Evernote}:
	      
	      Zmiana kierunku rozwoju aplikacji, wejście na rynek aplikacji mobilnych doprowadziła do ogromnego wzrostu popularności Evernote,
	      co oddaje ogromny potencjał aplikacji do notowania, będących zawsze "pod ręką" użytkowników.
	      Inne aplikacje również podążają za tym trendem.
	\item \textbf{2013 - Notion}:
	      
	      Znaczącą innowację przynosi Notion, oferując niespotykanie szeroki wachlarz możliwości.
	      O ile sama aplikacja charakteryzuje się stosunkową prostotą, o tyle możliwości, jakie oferuje użytkownikowi — są bardzo liczne
	      — jest to swoista "piaskownica danych", dowolnie organizowanych i personalizowanych przez użytkownika, przy użyciu notatników,
	      kalendarzy i tabel.
	\item \textbf{2020 - Obsidian, Logseq}:
	      
	      Kolejnym przełomowym krokiem rozwoju jest Obsidian. Jest on jeszcze prostszy w obsłudze w porównaniu z Notion,
	      gdyż opiera się o pliki markdown, jednak użytkownik może go dowolnie dostosować do swych potrzeb m.in. za sprawą pluginów.
	      W ślad za nim podąża Logseq, oprogramowanie Open-Source, które opiera się grafy i listy punktowane.
	\item \textbf{Współczesność}:
	      
	      Na przykładzie najnowszych aplikacji, widoczna jest tendencja, oddawać w ręce użytkowników prostych narzędzi,
	      które mogą być w możliwie szeroki sposób wykorzystane przez użytkownika — zgodnie z jego zamysłem i stylem pracy.
\end{itemize}
\textbf{Wnioski}
Jak więc wynika z powyższych przykładów, istnieje zauważalny potencjał, w stworzeniu aplikacji-notatnika ukierunkowanego na potrzeby muzyków.
W założeniach musiałoby być to narzędzie mobilne, proste i oferujące użytkownikowi dużą dowolność w sposobach notacji utworu.

\subsubsection{Badanie rynku}
\ldots

\subsubsection{Porównanie dostępnych rozwiązań}
Brak jest na rynku oprogramowania, które mogłoby wesprzeć proces wstępnej kompozycji i aranżacji utworów muzycznych.
Istnieją rozwiązania desktopowe, pozwalające tworzyć notatki i załączać do nich pliki, takie jak Obsidian i Logseq.
O ile są one rozbudowanymi narzędziami, nie są ukierunkowane na muzyków i same w sobie nie oferują kluczowych funkcji,
które pozwoliłyby je wykorzystać jako „muzyczny notatnik”.
W przypadku aplikacji mobilnych można wyróżnić przynajmniej dwie odrębne grupy:
\begin{itemize}
	\item aplikacje do tworzenia notatek — często pozwalające zapisać pliki dźwiękowe (np. Oferowane przez większość systemów „Notatki”);
	\item proste aplikacje do pisania nut lub akordów;
\end{itemize}
Brak jednak aplikacji, która umożliwiałaby połączyć te zastosowania.
Istnieje natomiast kilka rozwiązań wspomagających bardziej zaawansowane stadia rozwoju kompozycji,
doskonałym przykładem byłby tu GarageBand od Apple.
Jest to już jednak przede wszystkim DAW, umożliwiający w zasadzie już produkcję muzyki i nie oferuje narzędzi wspierających wczesnych,
konceptualnych faz rozwoju utworu.
Proponowana w tej pracy aplikacja czerpałaby inspiracje również z innych programów i serwisów.
Przykładem może być tu serwis streamingowy Soundcloud, wizualizujący falę dźwiękową pliku muzycznego, co ułatwia orientację w jego przebiegu.

\subsection{Identyfikacja aktorów}
Osoby zajmujące się muzyką, pracujące nad tworzeniem, aranżacją utworów.
Celem pracy jest zaprojektowanie aplikacji wspomagającej proces aranżacji utworów muzycznych.
Ma ona służyć za narzędzie umożliwiające zapisywanie szkiców melodii, tekstów oraz notatek i wskazówek do utworów.
Wspomniane funkcjonalności okażą się zatem użyteczne dla aranżerów, jak również kompozytorów i wykonawców.
Nie jest brany tu pod uwagę profesjonalizm użytkowników, raczej program ma odpowiadać potrzebom tych,
których proces twórczy przebiega nieoczekiwanie — w takich sytuacjach, podręczny „notatnik” muzyczny okaże się szczególnie przydatny.

\subsection{Identyfikacja problemów rozwiązywanych przez proponowaną aplikację}
\begin{itemize}
	\item Aranżer  ma kilka pomysłów na pewien jego fragment, ale jeszcze nie wie, który z nich wykorzysta:
	      może wszystkie je zapisać oraz opisać słownie, do późniejszej weryfikacji.
	\item Muzyk uczestniczy w próbie zespołu, dyrygent proponuje alternatywną linię melodyczną:
	      korzystając z aplikacji, muzyk może szybko i wygodnie zapisać ją w formie nutowej.
	\item Uczeń otrzymuje wskazówki od nauczyciela, dotyczące sposobu wykonania pewnego przebiegu w utworze:
	      korzystając z aplikacji, może zapisać przebieg jako nagranie audio, opisać go tekstem.
	\item Kompozytor tworzy nowy utwór, ma kilka fragmentów melodycznych, które chciałby wykorzystać:
	      może je zapisać i dowolnie zorganizować wizualnie (przesuwając, ukrywając elementy na tablicy).
	\item Muzyk nie ma przy sobie nut, otrzymuje zdjęcie od współczłonka zespołu:
	      może je zapisać w aplikacji, aby mieć do nich łatwy dostęp oraz odnotowywać na bieżąco istotne uwagi bezpośrednio
	      pod zapisem nutowym.
\end{itemize}
\subsection{Założenia projektowe}

\subsubsection{Wymagania jakościowe}
Biorąc pod uwagę charakter problemu oraz czerpiąc inspirację z pokrewnych współczesnych rozwiązań,
zdecydowano się na rozwiązanie będące aplikacją mobilną.
Jest to szczególnie istotne, z uwagi na dostępność i wygodę.
Telefony komórkowe stanowią w dzisiejszych czasach narzędzie uniwersalne, będące zawsze w zasięgu ręki [c.n.]
stąd stanowią doskonałą platformę dla aplikacji, mającej odpowiadać jako notatnik na \enquote{potrzebę chwili}.
Zaletę tę docenią także instrumentaliści — nie musząc odchodzić od instrumentu, by zapisać cenną i ulotną muzyczną myśl,
co zwykle ma miejsce przy pracy z programami komputerowymi, często również przy korzystaniu z nie-cyfrowych metod
— na przykład zeszytów.

Kolejną istotną cechą takiej aplikacji musi być jej niezależność od połączenia sieciowego.
Aplikacja powinna móc działać w pełni lokalnie — bez dostępu do sieci.
O ile rozwiązania sieciowe przynoszą wiele potencjalnie przydatnych funkcjonalności,
takich jak współdzielenie danych między urządzeniami,
o tyle (pomijając kwestię dodatkowego skomplikowania architektury programu)
sama aplikacja nie powinna być od nich zależna, z uwagi na warunki pracy muzyków.
Często sale koncertowe lub sale prób z uwagi na architekturę budynku mogą uniemożliwiać całkowicie
bądź w znaczącym stopniu połączenie sieciowe [c.n.]; również w trasie, czy podczas pobytu w obcym kraju
— internet bywa zawodny.
Stąd, aby zapewnić niezawodność działania aplikacji bez względu na warunki — musi ona działać bez dostępu do internetu.

W kwestii zasady działania — jako notatnik — aplikacja powinna stanowić swoiste środowisko,
oddając w ręce użytkownika decyzje o tym, jak zostanie ono wykorzystane.
W ten sposób jest w stanie odpowiedzieć na personalne potrzeby artysty, dostosować się do jego technik i charakteru pracy,
jak pusta karta o wirtualnie nieograniczonym potencjale.
Taki charakter przywodzi na myśl \enquote{piaskownicę}, jako odpowiedź na środowisko pracy kreatywnej [c.n.].

Idąc dalej tym tokiem, należałoby również nakreślić pewne właściwości samego interfejsu użytkownika.
Aby nie stanowił przeszkody w pracy, lecz współgrał z użytkownikiem, musi cechować się przejrzystością,
intuicyjnością oraz prostotą.
Przytoczone wcześniej serwisy [c.n.] mają te cechy, zatem należy przyjąć je i w tej kwestii za inspirację.
Aby osiągnąć zamierzony cel, w procesie projektowania interfejsu przyjęto także za ideę kluczową minimalizm;
słuszność tej decyzji mogą potwierdzić współczesne badania \cite{minimalism}.
Oprócz intuicyjności i wygody przekłada się to również na wysoki stopień interaktywności interfejsu — zawierając tylko to,
co konieczne można pozwolić, by każdy widoczny element umożliwiał określoną funkcjonalność
i odpowiadał na akcje użytkownika.

Możemy stąd wyciągnąć kluczowe wymagania jakościowe aplikacji:
\begin{itemize}
	\item mobilność
	\item niezawodność
	\item niezależność od warunków użytkowania
	\item prostota
	\item przejrzystość
	\item intuicyjność
	\item interaktywność
\end{itemize}

\subsection{Wymagania funkcjonalne}
\label{subsec:wymagania-funkcjonalne}
Najszerzej rzecz ujmując, aplikacja ma umożliwić użytkownikowi notowanie muzycznych koncepcji aranżacji utworów.
Należałoby tu odnieść się do wcześniejszego fragmentu pracy, ukazującej zróżnicowanie notacji muzycznej.
Aby pokryć możliwie szerokie spektrum praktyk zapisywania muzyki,
zdecydowano się na wyszczególnienie następujących formatów:
\begin{itemize}
	\item Nutowy — jako podstawowa i uniwersalna forma zapisu dźwięku.
	\item Tekstowy — jako medium pozwalające opisywać dźwięk, jak również zapisywać teksty utworów, chwyty gitarowe.
	\item Dźwiękowy — jako nagranie, umożliwia przechwycenie i zapis brzmienia.
	\item Wizualny — jako zdjęcia, przykładowo nut, ustawienia zespołu podczas prób.
\end{itemize}

Częstą praktyką wśród muzyków jest również notowanie wskazówek wykonawczych (na przykład artykulacyjnych)
na stronach z nutami.
Konieczne jest zatem umożliwienie użytkownikowi tworzenie podobnych adnotacji, jako komentarzy przy nutach,
bądź między liniami tekstu.
Aplikacja musi także umożliwiać określoną metodą organizacji tworzonych przez użytkownika zbiorów notatek,
umożliwiając łatwą ich lokalizację.
Dodatkowymi zaletami byłaby również możliwość wizualnego rozróżniania zbiorów notatek,
co może zostać osiągnięte przez (\ldots).

Aplikacja musi zatem umożliwiać:
\begin{itemize}
	\item Wyświetlanie, edycję oraz zapis notatek nutowych.
	\item Wyświetlanie, edycję oraz zapis notatek tekstowych.
	\item Dodawanie komentarzy do fragmentów notatek nutowych oraz tekstowych.
	\item Nagrywanie, zapis oraz wizualizację dźwięku.
	\item Wyświetlanie zdjęć jako notatek.
	\item Ustawianie zdjęć jako okładek zbiorów notatek.
	\item Ustawianie słów kluczowych jako metody identyfikacji zbiorów notatek.
	\item Wyszukiwanie zbiorów notatek po nazwie oraz po słowach kluczowych.
\end{itemize}
\newpage

\addcontentsline{toc}{section}{Projekt aplikacji wspomagającej aranżacje i usługi muzyczne}

\section*{Projekt aplikacji wspomagającej aranżacje i usługi muzyczne}
\newpage

\addcontentsline{toc}{section}{Analiza otrzymanych wyników i wnioski}
\section*{Analiza otrzymanych wyników i wnioski}
\newpage

\addcontentsline{toc}{section}{Bibliografia}
\printbibliography
\newpage

\addcontentsline{toc}{section}{Spis rysunków}
\listoffigures
\newpage

\addcontentsline{toc}{section}{Spis tabel}
\listoftables
\newpage

\end{document}
