\documentclass[12pt]{article}

\usepackage[polish]{babel}
\usepackage{csquotes}
\usepackage[a4paper,top=2.5cm,bottom=2.5cm,left=3cm,right=2cm]{geometry}
\usepackage{indentfirst}
\usepackage{lipsum}
\usepackage{amsmath}
\usepackage{graphicx}
\usepackage[colorlinks=true, allcolors=blue]{hyperref}
\usepackage{float}
\usepackage{fontspec}
\setmainfont{Times New Roman}
\setlength{\parindent}{1.25cm}
\linespread{1.5}
\setlength{\parskip}{6pt}
\usepackage[sorting=nyt, backend=biber]{biblatex}
\renewcommand*{\newunitpunct}{\addcomma\space}
\addbibresource{bibliography.bib}
\hypersetup{linkcolor=black}
\usepackage[all]{hypcap}
\addto\captionspolish{\renewcommand{\figurename}{Rys.}}
\addto\captionspolish{\renewcommand{\tablename}{Tab.}}
\renewcommand{\tableautorefname}{Tab.}
\renewcommand{\figureautorefname}{Rys.}
\renewcommand{\listfigurename}{Spis rysunków}
\newcommand{\image}[3][16cm]{
    \begin{figure}[H]
        \centering
        \includegraphics[width=#1]{media/#2}
        \caption{#3}
        \label{fig:2}
    \end{figure}
}
\usepackage{caption}
\captionsetup[figure]{labelsep=period}
\renewcommand{\thesection}{\arabic{section}.}
\renewcommand{\thesubsection}{\thesection\arabic{subsection}.}
\renewcommand{\thesubsubsection}{\thesubsection\arabic{subsubsection}.}
\usepackage{blindtext}
\usepackage{tocbasic}
\usepackage{capt-of}
\DeclareTOCStyleEntry[entrynumberformat=\adddot]{tocline}{figure}
\newcommand*{\adddot}[1]{#1\unskip.\hfil}
\begin{document}

\newpage
\clearpage
\tableofcontents


\newpage
\addcontentsline{toc}{section}{Słownik pojęć}
\section*{Słownik pojęć}
Celem przybliżenia konceptualnego projektu aplikacji, wykorzystano szereg pojęć, odnoszących się do poszczególnych jej
elementów (w nawiasach umieszczone zostały ich angielskie odpowiedniki użyte w kodzie aplikacji):
\begin{enumerate}
	\item \textit{strona} (Page) — pojedyncza notatka w ramach zbioru
	\item \textit{zeszyt} (Book) — zbiór notatek dotyczących jednego utworu, od \enquote{zeszytu do nut} (eng. \enquote{Music Book})
	\item \textit{biblioteka} (Library) — kolekcja zbiorów notatek użytkownika
	\item \textit{warsztat} (Workshop) - ekran edycji notatek
\end{enumerate}

\newpage
\addcontentsline{toc}{section}{Wstęp}
\section*{Wstęp}
Współcześne rozwiązania programistyczne wspierają działanie wielu dziedzin życia.
Udział aplikacji mobilnych wśród innych rodzajów oprogramowania staje się coraz bardziej znaczący,
wraz ze wzrostem możliwości telefonów komórkowych.
Atut mobilności jest wysoce ceniony, szczególnie w nieprzewidywalnych sytuacjach życia codziennego
— zapewniając dostępność, a co za tym idzie — wygodę, niezależną od okoliczności.

Jednocześnie jest wiele dziedzin pracy profesjonalnej, w których zdecydowanie dominują aplikacje na komputery osobiste.
Na potrzeby muzyków dostępne są obszerne ekosystemy takich aplikacji, których celem jest wspomaganie procesu powstawania utworów.
Dziedzina ta — jako dziedzina sztuki — często charakteryzuje się jednak dużą nieprzewidywalnością.
Procesy kreatywne artystów mogą przybierać nieuporządkowane struktury, przynoszące mnogość koncepcji i planów.
Wykorzystywane przez muzyków narzędzia programistyczne zdają się często charakteryzować nastawieniem na pracę
z koncepcją już \textit{stworzoną i określoną}.
Przypadek procesu aranżacji — opracowań utworów — można w tym świetle uznać za szczególny, jako ogniwo pomiędzy kompozycją
a interpretacją utworu. Wymagana jest organizacja i dyscyplina, by szkice koncepcji melodii przerodziły się w kompletne
dzieło muzyczne. Tkwi stąd potencjał w projekcie aplikacji — jako mobilnego narzędzia pozwalającego uporządkować i określić
poszczególne produkty procesu kreatywnego, nadając koncepcji wyrazisty kształt.

\newpage
\addcontentsline{toc}{section}{Cel i zakres pracy}
\section*{Cel i zakres pracy}
Celem pracy jest stworzenie aplikacji mobilnej, która umożliwi zapis i organizację koncepcji utworów muzycznych podczas procesu aranżacji.
Aplikacja ma stanowić narzędzie o prostej obsłudze, oferujące użytkownikowi dowolność w wyborze sposobu notacji muzycznej utworu.
Jej zadaniem jest \enquote{cyfrowa emulacja zeszytu do nut} jako rodzaju \textit{muzycznego notesu} oraz rozszerzenie jego możliwości: nie
ograniczając się jedynie do zapisu nutowego i tekstowego, lecz wykorzystując również multimedia — jako nośniki informacji o dźwięku.
Unikatowość projektu opiera się o ukierunkowanie funkcjonalności aplikacji na wspomaganie części procesów pomijanych przez współczesnie
dostępne oprogramowanie.

\newpage
\section{Studium w zakresie problematyki procesu aranżacji muzycznej}
\subsection{Definicja problematyki procesu aranżacji}
Aranżacją w ujęciu muzycznym jest
\enquote{opracowanie utworu muzycznego w nowym stylu bez zasadniczych zmian linii melodycznej}.
Stoi on z tego względu konceptualnie pomiędzy procesami kompozycji oraz interpretacji utworu
i w efekcie współdzieli częściowo problematykę obu tych etapów, jako proces techniczny oraz twórczy.
Jakość aranżacji zależy od znajomości teorii muzyki, technik kompozytorskich, jak również doświadczenia w pracy z muzyką.
Podczas tego procesu, muzyk nadaje utworowi swoisty kształt. Jest to efekt dynamicznej i zmiennej pracy kreatywnej, podczas
której wyłania się wiele muzycznych koncepcji, wymagających uporządkowania i selekcji.

\subsection{Rys historyczny muzyki klasycznej}
Aby zrozumieć charakter procesu aranżacji, należy przyjrzeć się rozwojowi muzyki na przestrzeni epok.
Szczególnie ważne w kontekście omawianego zagadnienia jest tu ewolucja technik twórczych, jak również samej
notacji muzycznej. Ukazują one ogromne zróżnicowanie podejść i sposobów pracy z muzyką.

\subsubsection{Rozwój technik procesu twórczego muzyki}
Na początku należałoby przytoczyć charakterystykę epok w historii muzyki, z naciskiem na charakter krajobrazu muzycznego
poszczególnych epok i tendencji popularnych wśród twórców.
\begin{itemize}
	\item \textbf{Średniowiecze}:

	      Proces twórczy muzyków epoki średniowiecza był głęboko osadzony w liturgicznym kontekście religijnym.
	      Znaczącą rolę ogdrywał chorał gregoriański jako podstawa tworzonych utworów.
	      Kompozytorzy tego okresu opierali się na zawartych w nim melodiach, wykorzystując techniki takie jak organum,
	      melizmaty oraz schematy rytmiczne, urozmajcając ich brzmienie.
	      Organum polegało na dodawaniu nowych dźwięków do melodii chorału, co pozwalało tworzyć wielowarstwowe melodie;
	      melizmaty stosowane były jako melodyczne ozdobniki. Kluczową rolę odgrywał tekst utworów.
	\item \textbf{Renesans}:

	      Punktem przełomowym dla epoki renesansu była polifonia — czyli nakładanie na siebie równoważnych głosów o
	      odmiennych liniach melodycznych.
	      Szczególną popularnością cieszyła się wówczas technika imitacji — naśladownictwa.
	      Zaczęto wówczas stosować także bardziej skomplikowane rytmy, na przykład — opierające się na trójdzielności
	      oraz nowe skale muzyczne.
	\item \textbf{Barok}:

	      Twórczość muzyków epoki baroku charakteryzowała się rozbudowaną ornamentacją.
	      Popularne stało się basso continuo, jako metoda prowadzenia akompaniamentu, oraz technika kontrapunktu,
	      w której kontrastujące melodie tworzyły złożone struktury dźwiękowe.
	      Kontynuowano również eksperymenty z harmonią i dynamiką.
	\item \textbf{Klasycyzm}:

	      W okresie klasycyzmu muzyczny proces twórczy cechowała kontrastująca z barokiem precyzja i klarowność.
	      Zaczęto stosować ścisłe formy — jak forma sonatowa — określające strukturę, tempo,
	      charakter i układ poszczególnych części utworów.
	      Wykorzystywane były powtarzalne motywy melodyczne; dążono do równowagi między poszczególnymi elementami składowymi utworów.
	\item \textbf{Romantyzm}:

	      W podobnym kontraście epoka romantyzmu przynosi duże oswobodzenie wyrazowe muzyki.
	      Kluczowym elementem motywującym twórczość staje się introspekcja, stąd utwory tego okresu nasycone są indywidualizmem
	      i wyrazistością emocjonalną.
	      Wciąż stosowane są pewne wzorce — gatunki, jak sonata czy fantazja, ale nie są one już tak ściśle określone,
	      stąd pojawiają się zróżnicowane eksperymenty kompozytorskie z nietypową harmonią, dynamicznymi kontrastami i rozbudowanymi
	      frazami melodycznymi.
	\item \textbf{XX wiek}:

	      W XX wieku obserwowane jest szczególne bogactwo i różnorodność podejść do kompozycji.
	      Pojawiają się tu zróżnicowane nurty, między innymi — impresjonizmu, ekspresjonizmu, neoklasycyzmu.
	      Muzyka staje się \enquote{produktem} licznych koncepcji i metod tworzenia — jak aleatoryzm, opierający
	      się o losowość lub serializm, kierowany przetworzeniami matematycznymi; a także idei — tu możnaby przytoczyć minimalizm.
	      Epoka ta przynosi jeszcze jeden przełom — komputery i ich zastosowanie w muzyce, jako narzędzia twórcze, nośniki i instrumenty.
\end{itemize}

\subsubsection{Notacja muzyczna}
Na przestrzeni epok kluczowym nośnikiem pozwalającym na utrwalenie melodii była notacja.
Ona również ewoluowała i zmieniała się, dostosowując się do zaawansowania rozwoju muzyki oraz potrzeb artystów.
Zauważalne jest tu zróżnicowanie konceptualne zapisu melodii.
\begin{itemize}
	\item \textbf{Starożytność}:

	      Najstarsza forma notacji muzycznej przypisywana jest Babilończykom (1400 rok p.n.e.), opiera się o zapis klinowy.
	      Podobnie późniejsze odkrycia — pochodzące ze Starożytnej Grecji — również opierają się o alfabet,
	      a także symbole określające długość dźwięku.
	\item \textbf{Średniowiecze}:

	      Pojawia się wówczas notacja dedykowana stricte muzyce — neumy. Pierwsza notacja tego typu wywodzi się z Imperium Bizantyjskiego,
	      jednak pod tym pojęciem rozumie się przede wszystkim europejską notację wykorzystywaną początkowo w chorale
	      gregoriańskim — do określania wysokości dźwięków oraz ich czasu trwania.
	      Epoka ta przynosi również specyficzny rodzaj nazewnictwa dźwięków — solmizację.
	\item \textbf{Renesans}:

	      Powstają podwaliny współczesnej, ogólnej notacji nutowej: zaczynają być stosowane nuty oraz symbole
	      — jak na przykład klucze.
	      Na przestrzeni kolejnych epok zapis jest rozwijany, umożliwiając bardzo dokładne określanie zamierzonego przez twórcę przebiegu
	      melodycznego.
	      Jednocześnie pojawiają się tabulatury — zapis ukierunkowany pod konkretny instrument, dostosowany pod jego charakterystykę —
	      przykładowo organy, lutnie itd.
	\item \textbf{Barok}:

	      Zaczyna być stosowany bas cyfrowany.
	      Jest to notacja numeryczna określające harmonię akompaniamentu utworów.
	      Rola ta przypadała zwykle konkretnym instrumentom — jak na przykład klawesynowi.
	      Zasada działania tego rodzaju notacji opierała się o akordy i relacje interwałowe pomiędzy poszczególnymi dźwiękami.
	\item \textbf{Klasycyzm i romantyzm}:

	      Epoki przynoszą dalszy rozwój standaryzowanego zapisu nutowego, celem oznaczenia ścisłej artykulacji, dynamiki i agogiki
	      dla wykonawców utworów.
	\item \textbf{XX wiek}:

	      Muzycy zaczynają eksperymentować z zapisem muzycznym, zupełnie dostosowując go do swoich potrzeb, tworząc warstwy abstrakcji,
	      własne \enquote{paradygmaty}, odwzorowujące eksperymentalność tworzonej przez nich muzyki.
	\item \textbf{Współczesność}:

	      Współcześnie korzysta się z wielu form zapisu muzycznego. Kluczową rolę odgrywa zapis nutowy,
	      lecz niektórzy artyści wciąż czasami sięgają po własne formy notacji, podobnie jak miało to miejsce w XX wieku.
	      Wielką popularnością cieszą się również tabulatury, zapis akordowy oraz wizualizacje komputerowe, z uwagi ta ich przystępność.
	      Jednocześnie w pewnych okolicznościach wciąż wykorzystywane są starsze metody zapisu — sięgające nawet do neum.
\end{itemize}

\subsubsection{Podsumowanie}
Jak więc widać, współcześnie wykorzystuje się szeroką gamę technik i sposobów notacji muzycznej. Dostępność edukacji muzycznej
oraz instrumentów przekłada się na duże zróżnicowanie \enquote{warsztatów} artystów. W następstwie XX—wiecznych tendencji kompozytorskich
oraz przemian społeczno—kulturowych osiągnięta zostaje całkowita dowolność w sposobach pracy z muzyką.

\subsection{Badanie rynku oprogramowania wspierającego pracę muzyków}
\subsubsection{Programy komputerowe}
Praca twórcza muzyków od XX wieku jest również wspierana przez programy komputerowe. Można zaobserwować dynamikę postępu
w tej dziedzinie, analizując następujące po sobie nowatorskie rozwiązania programistyczne na przestrzeni lat:
\begin{itemize}
	\item \textbf{1951 - początki zastosowania komputerów w muzyce}:

	      W 1951 roku inżynier komputerowy Max Mathews stworzył pierwszy program komputerowy do generowania dźwięku.
	      Program ten działał na komputerze IBM 704 i otworzył drzwi do eksperymentów z syntezą dźwięku przy użyciu komputerów.
	      W późniejszych latach powstaje "MUSIC I" — tego samego twórcy — umożliwiający tworzenie muzyki.
	      Można określić ten moment jako początek muzyki komputerowej.
	\item \textbf{1980 - MIDI (Musical Instrument Digital Interface) i edycja dźwięku}:

	      Wprowadzenie standardu MIDI umożliwiło komunikację między różnymi instrumentami muzycznymi a komputerami,
	      co znacząco ułatwiło produkcję i edycję muzyki przy użyciu oprogramowania.
	      W latach 80' pojawiły się też programy takie jak SoundEdit i Sound Forge, umożliwiające edycję dźwięku.
	      Zyskały one popularność w dziedzinie produkcji muzycznej.
	\item \textbf{1990 - DAW (Digital Audio Workstation), syntezatory i samplowanie}:

	      Powstanie programów DAW, takich jak Pro Tools, Cubase, Logic Pro, Ableton Live i FL Studio,
	      umożliwiło profesjonalistom tworzenie, edycję i produkcję muzyki na komputerach osobistych.
	      DAW stały się kluczowym narzędziem w przemyśle muzycznym.
	      Pojawiło się również oprogramowanie, które umożliwia emulację syntezatorów analogowych oraz samplowanie dźwięków.
	\item \textbf{2000 — Rozszerzanie zastosowań programów komputerowych}:

	      Oprogramowanie takie jak Kontakt, Omnisphere czy Massive pozwoliło na tworzenie wirtualnych instrumentów i efektów
	      dźwiękowych, co znacząco rozszerzyło możliwości produkcji muzycznej.
	      Powstają programy automatyzujące procesy: Ludwig 3.0 — tworzący automatyczne aranżacje, Transcribe — transkrybujący
	      dźwięk na notację muzyczną, jak również cała gama mniejszych i większych projektów wspomagających muzyków w procesie
	      twórczym.
\end{itemize}

\subsubsection{Aplikacje mobilne}
Współcześnie, przy wzroście możliwości smartfonów należałoby oczekiwać również wzrostu ich zastosowań w dziedzinie
muzyki. Rynek jednak oferuje dość nieliczne takie rozwiązania — często nastawione na wykonywanie pojedycznych,
prostych czynności. Za przykład może posłużyć tu Maestro — umożliwiający tworzenie zapisu nutowego lub aplikacje
służące za metronomy.

Znaczący potencjał aplikacji mobilnych jest jednak widoczny na przykładzie GarageBand. Jest to w zasadzie pełne DAW,
pozwalające na tworzenie muzyki przy użyciu smartfona. Umożliwia korzystanie z syntezatorów, nagrywanie, edycję dźwięku,
jak również możliwość nakładania efektów dźwiękowych. Mimo to, w przypadku tego rodzaju pracy, preferowane są
z rozwiązania desktopowe, które — choćby z uwagi na rozmiar ekranu — oferują większy komfort.

\subsubsection{Podsumowanie}
Analiza programów i aplikacji stanowiących zestaw narzędzi muzyków na przestrzeni lat oraz współcześnie
odsłania niezapełnioną niszę: wciąż brakuje rozwiązań — tak wśród aplikacji mobilnych, jak i programów komputerowych —
które mogłoby wspomóc pracę twórczą w fazach powstawania koncepcji, podczas aranżacji utworów muzycznych. Dotychczasowe
rozwiązania pozwalają zapisywać ułożone utwory oraz poddawać je edycji; także umożliwiają generowanie aranżacji bądź całych
utworów. Jednak elementy procesu aranżacji dokonywanej przez muzyków, wiążące się z mnogością pomysłów i licznych zmian —
nieczęsto są wspierane przez rozwiązania programistyczne.

\subsection{Rozwiązania oferowane przez inne aplikacje niepowiązane z procesem aranżacji}
Aby odpowiedzieć na poruszony wyżej problem, konieczna jest analiza oprogramowania przynoszącego rozwiązania problemów
o podobnym charakterze. Najbliższym przykładem będzie tu oprogramowanie umożliwiające tworzenie notatek.
Na poniższych przykładach, można zaobserwować rozwój przełomowych — a współcześnie kluczowych — funkcjonalności oferowanych
przez aplikacje z tej kategorii:
\begin{itemize}
	\item \textbf{1987 - Lotus Agenda}:

	      Lotus Agenda, stworzony przez firmę Lotus Development Corporation oparty, był jednym z pierwszych programów do
	      zarządzania notatkami.
	      Umożliwiał użytkownikom tworzenie notatek, organizowanie ich w kategorie i przeszukiwanie.
	\item \textbf{2000 - EverNote}:

	      Evernote, założone przez Stepana Pachikova, stało się jednym z najbardziej rozpoznawalnych narzędzi do tworzenia
	      notatek.
	      Aplikacja pozwala na tworzenie notatek w różnych formatach, w tym tekst, obrazy, dźwięki,
	      oraz synchronizację ich między różnymi urządzeniami.
	\item \textbf{2003 - Microsoft OneNote}:

	      Microsoft OneNote (pierwornie OneNote 2003), zyskał popularność jako narzędzie do tworzenia notatek w systemie
	      Windows.
	      W kolejnych edycjach umożliwiał użytkownikom innowacyjne organizowanie notatek w ramach \enquote{tablicy}.
	\item \textbf{2008 - Evernote jako aplikacja mobilna}:

	      Zmiana kierunku rozwoju aplikacji, wejście na rynek aplikacji mobilnych doprowadziła do ogromnego wzrostu popularności Evernote,
	      co ukazało duży potencjał aplikacji do notowania, będących zawsze \enquote{pod ręką} użytkowników.
	      Inne aplikacje również podążają za tym trendem.
	\item \textbf{2013 - Notion}:

	      Znaczącą innowację przynosi Notion, oferując dotąd niespotykany wachlarz możliwości.
	      Stanowi kolejny krok w ewolucji aplikacji do notowania — jako tzw. personalna baza wiedzy (eng.
	      \enquote{Personal Knowledge Base})
	      O ile sama aplikacja charakteryzuje się stosunkową prostotą, o tyle — jest to swoista \enquote{piaskownica danych},
	      dowolnie organizowanych i personalizowanych przez użytkownika, przy użyciu notatników, kalendarzy i tabel.
	\item \textbf{2020 - Obsidian, Logseq}:

	      Kolejnym przełomowym krokiem rozwoju jest Obsidian. Charakteryzuje się większą prostotą od Notion,
	      gdyż działa w oparciu o pliki markdown, jednak użytkownik może go dowolnie dostosować do swych potrzeb m.in.
	      za sprawą
	      wtyczek. W ślad za nim podąża Logseq, oprogramowanie Open-Source, które opiera się grafy i listy punktowane.
\end{itemize}
Na przykładzie najnowszych aplikacji, widoczna jest tendencja, oddawania w ręce użytkowników prostych narzędzi,
które mogą być w możliwie szeroki sposób wykorzystane przez użytkownika — zgodnie z jego zamysłem i stylem pracy.
Należałoby w tym miejscu odnotować także kluczowe cechy współczesnych aplikacji do notowania: mobilność, elastyczność
i możliwość organizacji.
Wymienione aplikacje reprezentują także różne podejścia do wizualizacji \enquote{notatki}.
Może być to strona z tekstem — Obsidian, element na liście punktowanej — Logseq, karta w układzie kanban — Notion,
element umiejscowiony (w dowolnym położeniu na) tablicy — Microsoft OneNote itd. Warto tu jednak zwrócić uwagę,
że znacząca część współczesnych rozwiązań umożliwia również wizualną || przestrzenną organizację notatek.


\newpage
\section{Założenia projektowe}
\subsection{Identyfikacja aktorów}
Aktorem korzystającym z zaprojektowanej aplikacji jest użytkownik.
W założeniu jest to osoba zajmująca się muzyką, szczególnie: pracująca nad tworzeniem i aranżacją utworów.
Nie jest tu brany pod uwagę profesjonalizm użytkownika; program ma odpowiedzieć jako podręczny notatnik przede wszystkim
na potrzeby muzyków, których proces twórczy nie przebiega w sposób całkowicie usystematyzowany, bywając efektem chwili.

\subsection{Poglądowe scenariusze wspierania procesu aranżacji}
Poniżej przedstawiono potencjalne sytuacje problematyczne, jakie mogłaby rozwiązać projektowana aplikacja:
\begin{itemize}
	\item Aranżer ma kilka pomysłów na pewien jego fragment, ale jeszcze nie wie, który z nich wykorzysta:
	      może wszystkie je zapisać jako nagrania oraz opisać słowami, do późniejszej weryfikacji.
	\item Muzyk uczestniczy w próbie zespołu, dyrygent proponuje mu alternatywną linię melodyczną:
	      korzystając z aplikacji, muzyk może w łatwy sposób zapisać ją w formie nutowej.
	\item Aranżer pracuje nad utworem, ma kilka fragmentów melodycznych, które zamierza wykorzystać:
	      może dowolnie je zorganizować w aplikacji, ułatwiając ich lokalizację.
\end{itemize}

\subsection{Wymagania}
\subsubsection{Wymagania jakościowe}
Biorąc pod uwagę charakter problemu oraz czerpiąc inspirację z pokrewnych współczesnych rozwiązań,
zdecydowano się na rozwiązanie będące aplikacją mobilną.
Jest to szczególnie istotne, z uwagi na dostępność i wygodę.
Telefony komórkowe stanowią współcześnie narzędzie uniwersalne, będące zawsze w zasięgu ręki,
stąd są najbardziej przystępną platformę dla aplikacji, mającej odpowiadać jako notatnik na \enquote{potrzebę chwili}.
Zaletę tę docenią także instrumentaliści — nie musząc odchodzić od instrumentu, by zapisać muzyczną myśl,
co zwykle ma miejsce przy pracy z programami komputerowymi bądź przy korzystaniu z nie-cyfrowych metod
— na przykład zeszytów.

Kolejną cechą takiej aplikacji musi być jej niezależność od połączenia sieciowego.
Aplikacja powinna móc działać w pełni lokalnie — bez dostępu do sieci.
O ile rozwiązania sieciowe przynoszą wiele potencjalnie przydatnych funkcjonalności,
takich jak współdzielenie danych między urządzeniami,
o tyle (pomijając kwestię dodatkowego skomplikowania architektury programu)
sama aplikacja nie powinna być od nich zależna, z uwagi na warunki pracy muzyków.
Będąc w trasie, podczas pobytu na sali koncertowej, bądź nawet w salach prób — mogą mieć oni ograniczony dostęp do internetu,
z uwagi na lokalizację, bądź architekturę budynku.
Stąd, aby zapewnić niezawodność działania aplikacji bez względu na warunki — musi ona działać bez dostępu do internetu.

W kwestii zasady działania — jako notatnik — aplikacja powinna stanowić swoiste środowisko,
w którym użytkownik decyduje o tym, w jaki sposób je wykorzysta.
W ten sposób aplikacja może odpowiedzieć na personalne potrzeby artysty, dostosowując się do jego technik i charakteru pracy.
Taki charakter przywodzi na myśl \enquote{piaskownicę}, jako środowisko pracy kreatywnej.

Idąc dalej, należałoby również określić właściwości samego interfejsu użytkownika.
Aby nie stanowił przeszkody w pracy, lecz współgrał z użytkownikiem, musi cechować się przejrzystością,
intuicyjnością oraz prostotą.
Aby osiągnąć zamierzony cel, w procesie projektowania interfejsu przyjęto także za ideę kluczową minimalizm~\cite{minimalism}.
Oprócz intuicyjności i wygody przekłada się ona również na wysoki stopień interaktywności interfejsu — zawierając tylko to,
co konieczne można pozwolić, by każdy widoczny element umożliwiał określoną funkcjonalność i odpowiadał na akcje użytkownika.

Możemy stąd wyciągnąć kluczowe wymagania jakościowe aplikacji:
\begin{itemize}
	\item mobilność,
	\item niezawodność,
	\item niezależność od warunków użytkowania,
	\item prostota,
	\item przejrzystość,
	\item intuicyjność,
	\item interaktywność.
\end{itemize}

\subsubsection{Wymagania funkcjonalne}
Najszerzej rzecz ujmując, aplikacja ma umożliwić użytkownikowi notowanie muzycznych koncepcji aranżacji utworów.
Należałoby tu odnieść się do wcześniejszego fragmentu pracy, ukazującego zróżnicowanie notacji muzycznej.
Aby pokryć możliwie szerokie spektrum metod zapisywania muzyki, zdecydowano się wyszczególnić następujące formaty:
\begin{itemize}
	\item Nutowy — jako podstawowa i uniwersalna forma zapisu dźwięku.
	\item Tekstowy — jako medium pozwalające opisywać dźwięk, jak również zapisywać teksty utworów, chwyty gitarowe.
	\item Dźwiękowy — jako nagranie, umożliwia przechwycenie i zapis brzmienia.
	\item Wizualny — jako zdjęcia, przykładowo nut, ustawienia zespołu podczas prób.
\end{itemize}

Częstą praktyką wśród muzyków jest również notowanie wskazówek wykonawczych (na przykład artykulacyjnych)
na stronach z nutami.
Przydatne okazałoby się zatem umożliwienie użytkownikowi tworzenie podobnych adnotacji, jako komentarzy przy nutach,
bądź między liniami tekstu.

Aby umożliwić wyszukiwanie oraz lokalizację elementów, należy zapewnić możliwość ich organizacji.
Biorąc przykład z przytoczonych wcześniej przykładów aplikacji do tworzenia notatek, może mieć ona charakter wizualny,
opierający się na umiejscowieniu elementu na określonej przestrzeni, w odniesieniu do innych elementów.
Również pomocna może okazać się możliwość wizualnej modyfikacji samego elementu — wyróżniając go.
Skuteczna organizacja wymaga jednak również systematycznej struktury poza wymiarem wizualnym, która umożliwiłaby wyszukiwanie
elementów według określonych przez użytkownika grup.

Na podstawie powyższych założeń, można ująć wymagania funkcjonalne aplikacji jako:
\begin{itemize}
	\item zapisywanie i edycja notatek opisujących założenia aranżacyjne utworu jako:
	      \subitem nuty
	      \subitem tekst
	      \subitem nagranie
	      \subitem obraz
	\item opatrywanie komentarzami fragmentów notatek nutowych i tekstowych
	\item organizacja przestrzenna notatek
	\item organizacja zbiorów notatek poprzez nadawanie tytułów i tagów
	\item wizualne wyróżnianie zbiorów notatek przez dołączanie zdjęć
\end{itemize}

\subsection{Wykaz użytych technologii}
Celem zrealizowania projektu wykorzystano następujące kluczowe technologie:
\begin{itemize}
	\item TypeScript (5.1.6)

	      Język programowania rozwijany przez Microsoft, będący nadzbiorem języka JavaScript.
	      Dzięki systemowi typów TypeScript wspiera programistów w wykrywaniu błędów na etapie kompilacji,
	      co zwiększa niezawodność i utrzymanie kodu w projektach. Z tego względu, jak również z uwagi na jego znajomość
	      przez Autora pracy, został wykorzystany jako główny język projektu.
	\item Ionic (cli — 7.1.5)

	      Framework do budowy hybrydowych aplikacji mobilnych, korzystający z języków webowych takich jak HTML,
	      CSS i JavaScript bądź TypeScript.
	      Ionic CLI ułatwia tworzenie, testowanie i wdrażanie aplikacji mobilnych na różne platformy, na podstawie jednego,
	      wspólnego kodu źródłowego.
	      Wybrano tę technologię z uwagi na jej uniwersalność pomiędzy platformami, oferującą potencjalne dalsze kierunki
	      rozwoju projektu, jak również zaznajomienie Autora pracy z podobnymi technologiami.
	\item Capacitor (core — 5.4.0)

	      Narzędzie do budowy natywnych aplikacji mobilnych, przy użyciu technologii webowych.
	      Capacitor umożliwia dostęp do funkcji natywnych urządzeń jak na przykład aparatu czy mikrofonu.
	      Współgra on z frameworkiem Ionic, celem dostarczania aplikacji na urządzenia z systemami Android i IOS.
	\item Vue.js (3.2.45)

	      Progresywny framework JavaScript do budowy interfejsów użytkownika.
	      Vue.js umożliwia łatwe tworzenie interaktywnych interfejsów, dzięki deklaratywnemu podejściu do komponentów i
	      reaktywnemu systemowi danych. Był on tworzony z myślą o przystępności i intuicyjności
	      — i z tego względu został wybrany do wykorzystania w połączeniu z Ionic.
	\item Pinia (2.1.4)

	      Biblioteka do zarządzania stanem globalnym aplikacji. Stanowi następstwo Vuex store — jako dedykowana dla
	      Vue.js. Umożliwia ona również korzystanie z narzędzi deweloperskich pozwalających śledzić zmiany stanu aplikacji.
	\item Ionic Storage (4.0.0)

	      Moduł do przechowywania danych w aplikacjach Ionic. Umożliwia przechowywanie i pobieranie danych
	      w aplikacjach mobilnych, zapewniając prosty interfejs do manipulacji danymi, przy użyciu par klucz—wartość.
	\item Biblioteki JavaScript do obsługi notatek:
	      \subitem SortableJs — dla umożliwienia wygodnej organizacji przestrzennej,
	      \subitem Wavesurfer.js — wizualizujący przebieg nagrania w formie dźwiękowej,
	      \subitem ABCjs — renderujący notację nutową na podstawie tekstu.

	\item Środowiska i aplikacje programistyczne:
	      \subitem JetBrains Webstorm — główne środowisko programistyczne przygotowania projektu
	      \subitem GitHub oraz Git — celem realizacji kontroli wersji
	      \subitem Buddy CI/CD — jako narzędzie do Continous Delivery, umożliwiające automatyczne budowanie projektu do
	      postaci aplikacji mobilnej.
\end{itemize}

\newpage
\addcontentsline{toc}{section}{Część praktyczna}
\section*{Część praktyczna}
\section{Projekt aplikacji wspomagającej aranżacje muzyczne}
\subsection{Model encji bazodanowych}
Dane użytkownika, zapisywane w aplikacji można sklasyfikować jako forma \enquote{notatek}.
Zdecydowano się nadać im modułowy charakter, pozwalając w ten sposób na wykorzystywanie różnych typów notacji muzycznej
w ramach określonego zbioru, który sumarycznie może opisywać aranżowany utwór.
Wykorzystano stąd dwupoziomowy podział obiektów. Jako pierwszy poziom przyjęto koncepcję
\enquote{zeszytu do nut}, jako zbioru notatek,
co przynosi stosowną nazwę wykorzystaną w kodzie źródłowym \equote{book} (od ang.
music book — \textit{zeszyt do nut}).
Idąc dalej tym tokiem, przyjęto określenie \enquote{page} (z ang. — textit{strona}),
jako odnoszące się do poziomu drugiego — pojedynczej notatki użytkownika.
Stąd też zbiór zeszytów użytkownika określany jest pojęciem \enquote{library} (z ang. — textit{biblioteka}).
\subsubsection{Zeszyt}
\subsubsection{Książka}
\subsection{Serwisy zarządzania persystencją}
W aplikacji zastosowano bazę danych Ionic Storage. Jest to baza typu klucz-wartość. Aby umożliwić
zarządzanie obiektami zagnieżdżonymi — jako: \textit{zeszyt} zawierać może wiele \textit{stron} — przy zachowaniu
niezależności operacji na obiektach, zastosowano trzy serwisy:
\subsubsection{StorageWrapper}
Z uwagi na charakter kluczy i wartości bazy — jako dwóch ciągów znaków — obiekty zapisywane w bazie są
konwertowane do ciągów JSON. [...]
\textit{Opakowuje} bibliotekę Ionic Storage, udostępniając podstawowe metody zarządzania danymi. Konwertuje
obiekty na ciągi znaków JSON, stanowiące wartości bazy. Jako jedyny bezpośrednio korzysta z metod oferowanych
przez Ionic Storage.
Wykorzystuje funkcje serwisu StorageWrapper

\subsection{Serwisy zarządzania stanem aplikacji}
\subsection{Projekt interfejsu}
\subsection{Implementacja widoków i komponentów głównych}
\subsection{Implementacja funkcjonalności wspierających aranżację}
\subsection{Testowanie aplikacji}


\newpage
\section{Analiza otrzymanych wyników i wnioski}
Zrealizowany projekt aplikacji odpowiada na problematykę procesu aranżacji, stanowiąc unikatowe narzędzie wielofunkcyjne,
jako organizer koncepcji muzycznych. Możliwości oferowane przez aplikacje wspierają części procesu twórczego pomijane przez
dostępne współcześnie oprogramowanie. Będąc najbliżej warsztatu kreatywnego, jako pierwsze \enquote{płótno}
umożliwiająca odnotowywanie koncepcji, aplikacja dąży do stania się istotnym narzędziem muzyka.
W ramach projektu udało się zrealizować założone wymagania:
\begin{enumerate}
	\item Aplikacja jest mobilna, działając na platformie Android.
	\item Przechowuje dane w pełni lokalnie na urządzeniu, oferując pełny zakres funkcjonalności niezależnie od dostępu do internetu.
	\item Projekt interfejsu jest zgodny z założeniami, realizowany zgodnie z ideą minimalizmu i użyteczności.
	\item Aplikacja umożliwia zapis oraz rewizję pomysłów i koncepcji w formie notatek — textit{stron}.
	\item Tworzone przez użytkownika strony mogą być realizowane przy użyciu różnych rodzajów notacji oraz jako multimedia.
	\item Strony mogą być organizowane wizualnie || przestrzennie — w określonej kolejności, umożliwiając ukrywanie zawartości.
	\item Zeszyty, stanowiące zbiory stron, mogą organizowane przy użyciu słów kluczowych oraz przeszukiwane — zarówno po słowach kluczowych, jak i tytułach.
	\item Wybrane przez użytkownika zeszyty mogą być opatrzone okładką, ułatwiającą ich identyfikację.
\end{enumerate}

Jednocześnie dostępne funkcjonalności wciąż mogą być rozbudowywane w wielu kierunkach,
zwiększając możliwości aplikacji oraz swobodę użytkownika podczas procesu twórczego:
\begin{enumerate}
	\item Rozszerzenie funkcjonalności istniejących typów stron: umożliwiając odtworzenie zapisu nutowego, modyfikację nagrania dźwiękowego.
	\item Możliwość dodawania innych typów stron — na przykład załączników PDF.
	\item Umożliwienie eksportu i importu tworzonych zeszytów oraz mechanizmy kopii zapasowych.
	\item Architektura umożliwiająca połączenie z internetem celem synchronizacji danych pomiędzy urządzeniami.
\end{enumerate}

Pozostają jednak wciąż pewne kwestie wymagające weryfikacji. Chociaż nie zauważono istotnych błędów działania podczas użytkowania aplikacji,
okres jej testowania jest zbyt krótki, aby móc poświadczyć jej niezawodność.
Problematyczny może się okazać mechanizm persystencji,
mogąc powodować nieprzewidziane zdarzenia przy wyłączeniu aplikacji przed zakończeniem przebiegu funkcji realizujących zapis danych w bazie.
W takiej sytuacji można temu wciąż zaradzić, rozbudowując architekturę serwisów persystencji lub zmieniając metodę zapisu danych.
Praca aplikacji może również przebiegać zawodnie na urządzeniach starszych, z uwagi na spadek wydajności. Skomplikowana problematyka
optymalizacji wykracza jednak poza zakres tej pracy.

Kod źródłowy aplikacji umożliwiający jej budowę oraz zbudowany instalacyjnym APK zostały umieszczone na płycie dołączonej do pracy.
Są one również dostępne na publicznym repozytorium GitHub na gałęzi enquote{praca-dyplomowa} \cite{source}.

\newpage
\addcontentsline{toc}{section}{Bibliografia}
\printbibliography

\addcontentsline{toc}{section}{Spis rysunków}
\listoffigures

\end{document}
