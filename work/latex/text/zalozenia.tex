\subsection{Identyfikacja aktorów}
Aktorem korzystającym z zaprojektowanej aplikacji jest użytkownik.
W założeniu jest to osoba zajmująca się muzyką, szczególnie: pracująca nad tworzeniem i aranżacją utworów.
Nie jest tu brany pod uwagę profesjonalizm użytkownika; program ma odpowiedzieć jako podręczny notatnik przede wszystkim
na potrzeby muzyków, których proces twórczy nie przebiega w sposób całkowicie usystematyzowany, bywając efektem chwili.

\subsection{Poglądowe scenariusze wspierania procesu aranżacji}
Poniżej przedstawiono potencjalne sytuacje problematyczne, jakie mogłaby rozwiązać projektowana aplikacja:
\begin{itemize}
	\item Aranżer ma kilka pomysłów na pewien jego fragment, ale jeszcze nie wie, który z nich wykorzysta:
	      może wszystkie je zapisać jako nagrania oraz opisać słowami, do późniejszej weryfikacji.
	\item Muzyk uczestniczy w próbie zespołu, dyrygent proponuje mu alternatywną linię melodyczną:
	      korzystając z aplikacji, muzyk może w łatwy sposób zapisać ją w formie nutowej.
	\item Aranżer pracuje nad utworem, ma kilka fragmentów melodycznych, które zamierza wykorzystać:
	      może dowolnie je zorganizować w aplikacji, ułatwiając ich lokalizację.
\end{itemize}

\subsection{Wymagania}
\subsubsection{Wymagania jakościowe}
Biorąc pod uwagę charakter problemu oraz czerpiąc inspirację z pokrewnych współczesnych rozwiązań,
zdecydowano się na rozwiązanie będące aplikacją mobilną.
Jest to szczególnie istotne, z uwagi na dostępność i wygodę.
Telefony komórkowe stanowią współcześnie narzędzie uniwersalne, będące zawsze w zasięgu ręki,
stąd są najbardziej przystępną platformę dla aplikacji, mającej odpowiadać jako notatnik na \enquote{potrzebę chwili}.
Zaletę tę docenią także instrumentaliści — nie musząc odchodzić od instrumentu, by zapisać muzyczną myśl,
co zwykle ma miejsce przy pracy z programami komputerowymi bądź przy korzystaniu z nie-cyfrowych metod
— na przykład zeszytów.

Kolejną cechą takiej aplikacji musi być jej niezależność od połączenia sieciowego.
Aplikacja powinna móc działać w pełni lokalnie — bez dostępu do sieci.
O ile rozwiązania sieciowe przynoszą wiele potencjalnie przydatnych funkcjonalności,
takich jak współdzielenie danych między urządzeniami,
o tyle (pomijając kwestię dodatkowego skomplikowania architektury programu)
sama aplikacja nie powinna być od nich zależna, z uwagi na warunki pracy muzyków.
Będąc w trasie, podczas pobytu na sali koncertowej, bądź nawet w salach prób — mogą mieć oni ograniczony dostęp do internetu,
z uwagi na lokalizację, bądź architekturę budynku.
Stąd, aby zapewnić niezawodność działania aplikacji bez względu na warunki — musi ona działać bez dostępu do internetu.

W kwestii zasady działania — jako notatnik — aplikacja powinna stanowić swoiste środowisko,
w którym użytkownik decyduje o tym, w jaki sposób je wykorzysta.
W ten sposób aplikacja może odpowiedzieć na personalne potrzeby artysty, dostosowując się do jego technik i charakteru pracy.
Taki charakter przywodzi na myśl \enquote{piaskownicę}, jako środowisko pracy kreatywnej.

Idąc dalej, należałoby również określić właściwości samego interfejsu użytkownika.
Aby nie stanowił przeszkody w pracy, lecz współgrał z użytkownikiem, musi cechować się przejrzystością,
intuicyjnością oraz prostotą.
Aby osiągnąć zamierzony cel, w procesie projektowania interfejsu przyjęto także za ideę kluczową minimalizm~\cite{minimalism}.
Oprócz intuicyjności i wygody przekłada się ona również na wysoki stopień interaktywności interfejsu — zawierając tylko to,
co konieczne można pozwolić, by każdy widoczny element umożliwiał określoną funkcjonalność i odpowiadał na akcje użytkownika.

Możemy stąd wyciągnąć kluczowe wymagania jakościowe aplikacji:
\begin{itemize}
	\item mobilność,
	\item niezawodność,
	\item niezależność od warunków użytkowania,
	\item prostota,
	\item przejrzystość,
	\item intuicyjność,
	\item interaktywność.
\end{itemize}

\subsubsection{Wymagania funkcjonalne}
Najszerzej rzecz ujmując, aplikacja ma umożliwić użytkownikowi notowanie muzycznych koncepcji aranżacji utworów.
Należałoby tu odnieść się do wcześniejszego fragmentu pracy, ukazującego zróżnicowanie notacji muzycznej.
Aby pokryć możliwie szerokie spektrum metod zapisywania muzyki, zdecydowano się wyszczególnić następujące formaty:
\begin{itemize}
	\item Nutowy — jako podstawowa i uniwersalna forma zapisu dźwięku.
	\item Tekstowy — jako medium pozwalające opisywać dźwięk, jak również zapisywać teksty utworów, chwyty gitarowe.
	\item Dźwiękowy — jako nagranie, umożliwia przechwycenie i zapis brzmienia.
	\item Wizualny — jako zdjęcia, przykładowo nut, ustawienia zespołu podczas prób.
\end{itemize}

Częstą praktyką wśród muzyków jest również notowanie wskazówek wykonawczych (na przykład artykulacyjnych)
na stronach z nutami.
Przydatne okazałoby się zatem umożliwienie użytkownikowi tworzenie podobnych adnotacji, jako komentarzy przy nutach,
bądź między liniami tekstu.

Aby umożliwić wyszukiwanie oraz lokalizację elementów, należy zapewnić możliwość ich organizacji.
Biorąc przykład z przytoczonych wcześniej przykładów aplikacji do tworzenia notatek, może mieć ona charakter wizualny,
opierający się na umiejscowieniu elementu na określonej przestrzeni, w odniesieniu do innych elementów.
Również pomocna może okazać się możliwość wizualnej modyfikacji samego elementu — wyróżniając go.
Skuteczna organizacja wymaga jednak również systematycznej struktury poza wymiarem wizualnym, która umożliwiłaby wyszukiwanie
elementów według określonych przez użytkownika grup.

Na podstawie powyższych założeń, można ująć wymagania funkcjonalne aplikacji jako:
\begin{itemize}
	\item zapisywanie i edycja notatek opisujących założenia aranżacyjne utworu jako:
	      \subitem nuty
	      \subitem tekst
	      \subitem nagranie
	      \subitem obraz
	\item opatrywanie komentarzami fragmentów notatek nutowych i tekstowych
	\item organizacja przestrzenna notatek
	\item organizacja zbiorów notatek poprzez nadawanie tytułów i tagów
	\item wizualne wyróżnianie zbiorów notatek przez dołączanie zdjęć
\end{itemize}

\subsection{Wykaz użytych technologii}
Celem zrealizowania projektu wykorzystano następujące kluczowe technologie:
\begin{itemize}
	\item TypeScript (5.1.6)

	      Język programowania rozwijany przez Microsoft, będący nadzbiorem języka JavaScript.
	      Dzięki systemowi typów TypeScript wspiera programistów w wykrywaniu błędów na etapie kompilacji,
	      co zwiększa niezawodność i utrzymanie kodu w projektach. Z tego względu, jak również z uwagi na jego znajomość
	      przez Autora pracy, został wykorzystany jako główny język projektu.
	\item Ionic (cli — 7.1.5)

	      Framework do budowy hybrydowych aplikacji mobilnych, korzystający z języków webowych takich jak HTML,
	      CSS i JavaScript bądź TypeScript.
	      Ionic CLI ułatwia tworzenie, testowanie i wdrażanie aplikacji mobilnych na różne platformy, na podstawie jednego,
	      wspólnego kodu źródłowego.
	      Wybrano tę technologię z uwagi na jej uniwersalność pomiędzy platformami, oferującą potencjalne dalsze kierunki
	      rozwoju projektu, jak również zaznajomienie Autora pracy z podobnymi technologiami.
	\item Capacitor (core — 5.4.0)

	      Narzędzie do budowy natywnych aplikacji mobilnych, przy użyciu technologii webowych.
	      Capacitor umożliwia dostęp do funkcji natywnych urządzeń jak na przykład aparatu czy mikrofonu.
	      Współgra on z frameworkiem Ionic, celem dostarczania aplikacji na urządzenia z systemami Android i IOS.
	\item Vue.js (3.2.45)

	      Progresywny framework JavaScript do budowy interfejsów użytkownika.
	      Vue.js umożliwia łatwe tworzenie interaktywnych interfejsów, dzięki deklaratywnemu podejściu do komponentów i
	      reaktywnemu systemowi danych. Był on tworzony z myślą o przystępności i intuicyjności
	      — i z tego względu został wybrany do wykorzystania w połączeniu z Ionic.
	\item Pinia (2.1.4)

	      Biblioteka do zarządzania stanem globalnym aplikacji. Stanowi następstwo Vuex store — jako dedykowana dla
	      Vue.js. Umożliwia ona również korzystanie z narzędzi deweloperskich pozwalających śledzić zmiany stanu aplikacji.
	\item Ionic Storage (4.0.0)

	      Moduł do przechowywania danych w aplikacjach Ionic. Umożliwia przechowywanie i pobieranie danych
	      w aplikacjach mobilnych, zapewniając prosty interfejs do manipulacji danymi, przy użyciu par klucz—wartość.
	\item Biblioteki JavaScript do obsługi notatek:
	      \subitem SortableJs — dla umożliwienia wygodnej organizacji przestrzennej,
	      \subitem Wavesurfer.js — wizualizujący przebieg nagrania w formie dźwiękowej,
	      \subitem ABCjs — renderujący notację nutową na podstawie tekstu.

	\item Środowiska i aplikacje programistyczne:
	      \subitem JetBrains Webstorm — główne środowisko programistyczne przygotowania projektu
	      \subitem GitHub oraz Git — celem realizacji kontroli wersji
	      \subitem Buddy CI/CD — jako narzędzie do Continous Delivery, umożliwiające automatyczne budowanie projektu do
	      postaci aplikacji mobilnej.
\end{itemize}