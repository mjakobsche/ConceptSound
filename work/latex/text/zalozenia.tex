\subsection{Identyfikacja aktorów}
Aktorem korzystającym z zaprojektowanej aplikacji jest jej użytkownik.
W założeniu jest to osoba zajmująca się muzyką, szczególnie: pracująca nad tworzeniem i aranżacją utworów.
Nie jest tu brany pod uwagę profesjonalizm użytkownika; program ma odpowiedzieć jako podręczny notatnik przede wszystkim
na potrzeby muzyków, których proces twórczy nie przebiega w sposób całkowicie usystematyzowany, często jako \enquote{efekt chwili}.

\subsection{Poglądowe scenariusze wspierania procesu aranżacji}
Poniżej przedstawiono przykładowe sytuacje problematyczne, jakie mogłaby rozwiązać projektowana aplikacja:
\begin{enumerate}
	\item Aranżer ma kilka pomysłów na pewien fragment utworu, ale jeszcze nie wie, który z nich wykorzysta:
	      może wszystkie je zapisać jako nagrania oraz opisać słowami, do późniejszej weryfikacji.
	\item Muzyk uczestniczy w próbie zespołu, dyrygent proponuje mu alternatywną linię melodyczną:
	      korzystając z aplikacji, muzyk może zapisać ją w formie nutowej.
	\item Aranżer pracuje nad utworem, ma kilka fragmentów melodycznych, które zamierza wykorzystać:
	      może dowolnie je zorganizować w aplikacji, ułatwiając ich lokalizację.
\end{enumerate}

\subsection{Wymagania}
\subsubsection{Wymagania jakościowe}
Biorąc pod uwagę charakter problemu oraz czerpiąc inspirację z pokrewnych współczesnych rozwiązań,
zdecydowano się na rozwiązanie będące aplikacją mobilną.
Jest to szczególnie istotne, przez wzgląd na wygodę i dostępność.
Telefony komórkowe stanowią współcześnie narzędzie uniwersalne, będące zawsze w zasięgu ręki \cite{10},
stąd są najbardziej przystępną platformę dla aplikacji, umożliwiając realizację \enquote{podręcznego notatnika}.

Kolejną wymaganą cechą aplikacji jest jej niezależność od połączenia sieciowego.
O~ile rozwiązania sieciowe przynoszą wiele potencjalnie przydatnych funkcjonalności,
takich jak współdzielenie danych między urządzeniami,
o tyle (pomijając kwestię dodatkowego skomplikowania architektury programu)
sama aplikacja nie powinna być od nich zależna, z uwagi na warunki pracy muzyków.
Będąc w trasie, podczas pobytu na sali koncertowej, bądź nawet w salach prób — mogą mieć oni ograniczony dostęp do internetu,
z uwagi na lokalizację, bądź architekturę budynku [Wymagane źródło bibliograficzne].
Stąd, aby zapewnić niezawodność działania aplikacji bez względu na warunki, wymagana jest \textit{lokalna} praca aplikacji —
bez połączenia sieciowego.

W kwestii zasady działania — jako notatnik — aplikacja powinna stanowić swoiste \enquote{środowisko},
w którym użytkownik decyduje o tym, w jaki sposób je wykorzysta.
W ten sposób aplikacja może odpowiedzieć na personalne potrzeby artysty, dostosowując się do jego technik i charakteru pracy.

Aby interfejs aplikacji nie stanowił przeszkody w pracy, lecz współgrał z użytkownikiem, musi cechować się przejrzystością,
intuicyjnością oraz prostotą.
Przyjmując za ideę minimalizm, jako kierunek projektu interfejsu użytkownika, osiągany jest jednocześnie wysoki stopień
interaktywności — zawierając tylko to, co konieczne można pozwolić, by każdy widoczny element umożliwiał określoną
funkcjonalność i odpowiadał na akcje użytkownika.

Przedstawione powyżej założenia stanowią podstawę do sformułowania kluczowych wymagań jakościowych aplikacji:
\begin{enumerate}
	\item Mobilność.
	\item Niezawodność.
	\item Niezależność od warunków użytkowania.
	\item Prostota.
	\item Przejrzystość.
	\item Intuicyjność.
	\item Interaktywność.
\end{enumerate}

\subsubsection{Wymagania funkcjonalne}
W najszerszym ujęciu aplikacja ma umożliwić użytkownikowi notowanie muzycznych koncepcji aranżacji utworów.
Odnosząc się do zróżnicowania notacji muzycznej — aby pokryć możliwie szerokie spektrum metod zapisu muzyki,
wyszczególnione zostały formaty:
\begin{enumerate}
	\item Nutowy — jako podstawowa i uniwersalna forma zapisu dźwięku.
	\item Tekstowy — jako medium pozwalające opisywać dźwięk, jak również zapisywać teksty utworów, chwyty gitarowe.
	\item Dźwiękowy — jako nagranie, umożliwia przechwycenie i zapis brzmienia.
	\item Wizualny — jako zdjęcia, przykładowo nut, ustawienia zespołu podczas prób.
\end{enumerate}

Częstą praktyką wśród muzyków jest również notowanie wskazówek wykonawczych (na przykład artykulacyjnych)
na stronach z nutami.
Przydatne okazuje się stąd umożliwienie użytkownikowi tworzenie podobnych adnotacji, jako komentarzy przy nutach
lub między liniami tekstu.

Możliwość organizacji elementów umożliwia ich wyszukiwanie oraz lokalizację.
Biorąc przykład z przytoczonych wcześniej aplikacji do tworzenia notatek, może mieć ona charakter wizualny,
opierający się na umiejscowieniu elementu na określonej przestrzeni, w odniesieniu do innych elementów.
Także pomocna jest możliwość wizualnej modyfikacji samego elementu — wyróżniając go na tle innych.

Skuteczna organizacja wymaga jednak również systematycznej struktury — poza wymiarem wizualnym — celem umożliwienia wyszukiwania
elementów według określonych przez użytkownika grup. Wybrano w tym celu mechanizm tagowania (oznaczania słowami kluczowymi), jako bardziej
elastyczny od struktury drzewiastej (realizowanej na przykład jako \textit{system plików i folderów}). Umożliwia on przynależność pojedynczego
elementu do wielu grup.

Na podstawie powyższych założeń, można ująć wymagania funkcjonalne aplikacji jako:
\begin{enumerate}
	\item Zapisywanie i edycja notatek opisujących założenia aranżacyjne utworu jako: nuty, tekst, nagranie, obraz.
	\item Opatrywanie komentarzami fragmentów notatek nutowych i tekstowych.
	\item Organizacja przestrzenna notatek.
	\item Organizacja zbiorów notatek poprzez nadawanie tytułów i tagów.
	\item Wizualne wyróżnianie zbiorów notatek przez dołączanie zdjęć.
\end{enumerate}

\subsection{Wykaz użytych technologii}
Celem zrealizowania projektu wykorzystano następujące kluczowe technologie:
\begin{enumerate}
	\item Node.js (18.16.1) oraz npm (9.5.1)

	      Node.js to środowisko uruchomieniowe JavaScript na silniku V8 \cite{node}.
	      Narzędzie npm (Node Package Manager) jest integralną częścią ekosystemu Node.js,
	      zapewniając mechanizm zarządzania zależnościami oraz dostarczając modułów i pakietów,
	      co znacząco ułatwia proces tworzenia oraz zarządzanie projektem opartym na Node.js \cite{npm}.
	\item TypeScript (5.1.6)

	      Język programowania rozwijany przez Microsoft, będący nadzbiorem języka JavaScript.
	      Dzięki systemowi typów TypeScript wspiera programistów w wykrywaniu błędów na etapie kompilacji,
	      co zwiększa niezawodność i utrzymanie kodu w projektach \cite{ts}.
	      Z~tego względu, jak również z uwagi na jego znajomość
	      przez Autora pracy, wykorzystany jest jako główny język projektu.
	\item Ionic Framework (cli — 7.1.5)

	      Framework do budowy hybrydowych aplikacji mobilnych, korzystający z języków webowych takich jak HTML,
	      CSS i JavaScript bądź TypeScript.
	      Ionic CLI ułatwia tworzenie, testowanie i wdrażanie aplikacji mobilnych na różne platformy, na podstawie jednego,
	      wspólnego kodu źródłowego \cite{ionic}.
	      Wybrano tę technologię z uwagi na jej uniwersalność międzyplatformową, oferującą dalsze potencjalne kierunki
	      rozwoju projektu.
	\item Capacitor (core — 5.4.0)

	      Narzędzie do budowy natywnych aplikacji mobilnych, przy użyciu technologii webowych.
	      Capacitor umożliwia dostęp do funkcji natywnych urządzeń jak na przykład aparatu czy mikrofonu.
	      Współgra on z frameworkiem Ionic, celem dostarczania aplikacji na urządzenia z systemami Android i IOS \cite{capacitor}.
	\item Vue.js (3.2.45)

	      Progresywny framework JavaScript do budowy interfejsów użytkownika.
	      Vue.js umożliwia łatwe tworzenie interaktywnych interfejsów, dzięki deklaratywnemu podejściu do komponentów i
	      reaktywnemu systemowi danych. Tworzony jest z myślą o przystępności i intuicyjności \cite{vue}
	      — z tego względu został wybrany do wykorzystania, w połączeniu z~Ionic.
	\item Pinia (2.1.4)

	      Biblioteka do zarządzania stanem globalnym aplikacji. Stanowi następstwo Vuex store — jako dedykowana dla
	      Vue.js. Umożliwia ona również korzystanie z narzędzi deweloperskich pozwalających śledzić zmiany stanu aplikacji \cite{pinia}.
	\item Ionic Storage (4.0.0)

	      Umożliwia przechowywanie i pobieranie danych
	      w aplikacjach mobilnych, zapewniając prosty interfejs do manipulacji danymi, przy użyciu bazy klucz—wartość.
	      Jest aktywnie wspierany oraz zgodny z Ionic w wersji 7, co nie jest aktualnie oferowane inne biblioteki persystencji \cite{storage}.
	\item Biblioteki JavaScript do obsługi notatek:

	      Sortable — dla umożliwienia wygodnej organizacji przestrzennej \cite{sortablejs},
	      wavesurfer.js — wizualizujący przebieg nagrania w formie dźwiękowej \cite{wavesurfer},
	      abcjs — renderujący notację nutową na podstawie tekstu \cite{abcjs}.

	\item Środowiska i aplikacje programistyczne:
	      JetBrains WebStorm — główne środowisko programistyczne przygotowania projektu \cite{webstorm},
	      GitHub oraz Git — celem realizacji kontroli wersji \cite{github}\cite{git},
	      Buddy CI/CD — jako narzędzie do Continous Integration i Continous Delivery, umożliwiające automatyczne budowanie projektu do
	      postaci aplikacji mobilnej \cite{buddy}.
\end{enumerate}