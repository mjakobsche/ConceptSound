\subsection{Model encji bazodanowych}
Dane użytkownika, zapisywane w aplikacji można sklasyfikować jako forma \enquote{notatek}.
Zdecydowano się nadać im modułowy charakter, pozwalając w ten sposób na wykorzystywanie różnych typów notacji muzycznej
w ramach określonego zbioru, który sumarycznie może opisywać aranżowany utwór.
Wykorzystano stąd dwupoziomowy podział obiektów. Jako pierwszy poziom przyjęto koncepcję
\enquote{zeszytu do nut}, jako zbioru notatek,
co przynosi stosowną nazwę wykorzystaną w kodzie źródłowym \equote{book} (od ang.
music book — \textit{zeszyt do nut}).
Idąc dalej tym tokiem, przyjęto określenie \enquote{page} (z ang. — textit{strona}),
jako odnoszące się do poziomu drugiego — pojedynczej notatki użytkownika.
Stąd też zbiór zeszytów użytkownika określany jest pojęciem \enquote{library} (z ang. — textit{biblioteka}).
\subsubsection{Zeszyt}
\subsubsection{Książka}
\subsection{Serwisy zarządzania persystencją}
W aplikacji zastosowano bazę danych Ionic Storage. Jest to baza typu klucz-wartość. Aby umożliwić
zarządzanie obiektami zagnieżdżonymi — jako: \textit{zeszyt} zawierać może wiele \textit{stron} — przy zachowaniu
niezależności operacji na obiektach, zastosowano trzy serwisy:
\subsubsection{StorageWrapper}
Z uwagi na charakter kluczy i wartości bazy — jako dwóch ciągów znaków — obiekty zapisywane w bazie są
konwertowane do ciągów JSON. [...]
\textit{Opakowuje} bibliotekę Ionic Storage, udostępniając podstawowe metody zarządzania danymi. Konwertuje
obiekty na ciągi znaków JSON, stanowiące wartości bazy. Jako jedyny bezpośrednio korzysta z metod oferowanych
przez Ionic Storage.
Wykorzystuje funkcje serwisu StorageWrapper

\subsection{Serwisy zarządzania stanem aplikacji}
\subsection{Projekt interfejsu}
\subsection{Implementacja widoków i komponentów głównych}
\subsection{Implementacja funkcjonalności wspierających aranżację}
\subsection{Testowanie aplikacji}
