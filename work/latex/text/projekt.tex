\subsection{Architektura aplikacji}
Jako model architektury aplikacji przyjęto architekturę wartstwową. Pozwala ona odseparować zadania wykonywane przez
poszczególne komponenty aplikacji, ułatwiając utrzymanie kodu. Jest to osiągane poprzez rozgranizczenie zadań
realizowanych przez poszczególne warstwy. Architektura wartstwowa charakteryzuje się hierarchicznym układem komponentów,
co przekłada się na zwiększenie przejrzystości pozwalające na elastyczny rozwój aplikacji.
W ramach projektu zastosowano trzy warstwy podziału:
\begin{itemize}
	\item Warstwa dostępu do danych — odpowiadająca za komunikacje z bazą przy użyciu modelu encji.
	\item Warstwa logiki biznesowej — obejmująca szczególnie serwisy zarządzania stanem aplikacji.
	\item Warstwa prezentacji — realizowana przez komponenty Vue.
\end{itemize}
<diagram architektury>
\subsection{Warstwa dostępu do danych}
\subsubsection{Modele bazodanowe}
Dane użytkownika, zapisywane w aplikacji można sklasyfikować jako forma \enquote{notatek}.
Zdecydowano się nadać im modułowy charakter, pozwalając w ten sposób na wykorzystywanie różnych typów notacji muzycznej
w ramach określonego zbioru, który sumarycznie może opisywać aranżowany utwór.
Wykorzystano stąd dwupoziomowy podział obiektów. Jako pierwszy poziom przyjęto koncepcję
\enquote{zeszytu do nut}, jako zbioru notatek,
co przynosi stosowną nazwę wykorzystaną w kodzie źródłowym \enquote{book} (od ang.
music book — \textit{zeszyt do nut}).
<encja zeszytu>
Idąc dalej tym tokiem, przyjęto określenie \enquote{page} (z ang. — textit{strona}),
jako odnoszące się do poziomu drugiego — pojedynczej notatki użytkownika.
<encja książki>
Obie encje implementują prosty interfejs Entity, deklarujący pole id,
ułatwiając działanie serwisów zarządzających persystencją:
<interfejs entity>
Jako wyznacznik powiązań między encjami Book i Page, w bazie przechowywana jest dotatkowo tablica obiektów Index.
Z uwagi na jej nadrzędny charakter, nie implementuje ona od interfejsu Entity. Występuje w bazie pojedynczo, pod ustalonym
kluczem \enquote{INDEX}.
<encja index>
\subsubsection{Serwisy zarządzające persystencją}
W aplikacji zastosowano bazę danych Ionic Storage. Jest to baza typu klucz-wartość. Aby umożliwić
zarządzanie obiektami zagnieżdżonymi — jako: \textit{zeszyt} zawierać może wiele \textit{stron} — przy zachowaniu
niezależności operacji na obiektach, zastosowano trzy serwisy:
\begin{itemize}
	\item StorageWrapper

	      Z uwagi na charakter kluczy i wartości bazy — jako dwóch ciągów znaków — obiekty zapisywane w bazie są
	      konwertowane do ciągów JSON. StorageWrapper \textit{Opakowuje} bibliotekę Ionic Storage, udostępniając podstawowe
	      metody zarządzania danymi oraz konwertując przetwarzane dane na ciągi znaków JSON i odwrotnie.
	      Jako jedyny bezpośrednio korzysta z metod oferowanych przez Ionic Storage.
	\item PersistencyService

	      Z StorageWrappera korzysta explicite PersistencyService, wykorzystujący jego metody w kontekście konkretnych obiektów.
	      Zawiera logikę opartą o interfejs Entity, implementowany przez encje Book i Page. Odwołuje się do danych
	      serwisu Indexer, celem identyfikacji i lokalizacji obiektów zapisanych w bazie.
	\item Indexer

	      Poza encjami Book i Page przechowywana jest również tablica obiektów Index określająca
	      zależności między encjami. Stąd stan początkowy Indexera jest wczytywany z bazy przez PersistencyService
	      na początku działania aplikacji.
	      Informacje o zmianach są sygnalizowane przez serwisy zarządzania stanem aplikacji.
	      Odpowiadając na nie, serwis Indexer zapewnia zgodność tablicy Indexów ze stanem rzeczywistym
	      \textit{biblioteki} użytkownika. W pewnym stopniu emuluje on możliwości bazy relacyjnej.
\end{itemize}
Zależności między opisanymi serwisami obrazuje przedstawiony poniżej diagram:
<diagram persystencji>
\subsection{Warstwa logiki biznesowej}
\subsubsection{Serwisy zarządzające stanem aplikacji}
Warstwę logiki biznesowej stanowią serwisy zarządzania stanem aplikacji.
Wszystkie opierają się na bibliotece Pinia.
Umożliwia to funkcjonowanie niezależne od komponentów Vue aplikacji, udostępniając dla nich globalny stan,
do którego mogą się odwoływać. Serwisy zostały podzielone według domen:
\begin{itemize}
	\item
\end{itemize}
Serwisy zarządzania stanem komunikują się z serwisami odpowiadającymi za persystencję,
do odczytu i zapisu danych w pamięci trwałej. Odczytane obiekty w ramach bieżących potrzeb aplikacji, przyjmowane są
za \textit{stan} danego serwisu. Komponenty vue mogą dzięki temu z nich korzytać — także — modyfikując je, przy użyciu
funkcji udostepnianych przez serwisy. Obserwacja zmian globalnego stanu pozwala odzworować je, zapisując w bazie danych.



\subsection{Warstwa prezentacji}
\subsubsection{Projekt interfejsu}
\subsubsection{Implementacja widoków i komponentów głównych}
\subsubsection{Implementacja funkcjonalności wspierających aranżację}
\subsection{Testowanie aplikacji}
