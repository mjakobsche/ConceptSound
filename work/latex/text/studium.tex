\subsection{Definicja problematyki procesu aranżacji}
Aranżacją w ujęciu muzycznym jest
\enquote{opracowanie utworu muzycznego w nowym stylu bez zasadniczych zmian linii melodycznej}.
Stoi on z tego względu konceptualnie pomiędzy procesami kompozycji oraz interpretacji utworu
i w efekcie współdzieli częściowo problematykę obu tych etapów, jako proces techniczny oraz twórczy.
Jakość aranżacji zależy od znajomości teorii muzyki, technik kompozytorskich, jak również doświadczenia w pracy z muzyką.
Podczas tego procesu, muzyk nadaje utworowi swoisty kształt. Jest to efekt dynamicznej i zmiennej pracy kreatywnej, podczas
której wyłania się wiele muzycznych koncepcji, wymagających uporządkowania i selekcji.

\subsection{Rys historyczny muzyki klasycznej}
Aby zrozumieć charakter procesu aranżacji, należy przyjrzeć się rozwojowi muzyki na przestrzeni epok.
Szczególnie ważne w kontekście omawianego zagadnienia jest tu ewolucja technik twórczych, jak również samej
notacji muzycznej. Ukazują one ogromne zróżnicowanie podejść i sposobów pracy z muzyką.

\subsubsection{Rozwój technik procesu twórczego muzyki}
Na początku należałoby przytoczyć charakterystykę epok w historii muzyki, z naciskiem na charakter krajobrazu muzycznego
poszczególnych epok i tendencji popularnych wśród twórców.
\begin{itemize}
	\item \textbf{Średniowiecze}:

	      Proces twórczy muzyków epoki średniowiecza był głęboko osadzony w liturgicznym kontekście religijnym.
	      Znaczącą rolę ogdrywał chorał gregoriański jako podstawa tworzonych utworów.
	      Kompozytorzy tego okresu opierali się na zawartych w nim melodiach, wykorzystując techniki takie jak organum,
	      melizmaty oraz schematy rytmiczne, urozmajcając ich brzmienie.
	      Organum polegało na dodawaniu nowych dźwięków do melodii chorału, co pozwalało tworzyć wielowarstwowe melodie;
	      melizmaty stosowane były jako melodyczne ozdobniki. Kluczową rolę odgrywał tekst utworów.
	\item \textbf{Renesans}:

	      Punktem przełomowym dla epoki renesansu była polifonia — czyli nakładanie na siebie równoważnych głosów o
	      odmiennych liniach melodycznych.
	      Szczególną popularnością cieszyła się wówczas technika imitacji — naśladownictwa.
	      Zaczęto wówczas stosować także bardziej skomplikowane rytmy, na przykład — opierające się na trójdzielności
	      oraz nowe skale muzyczne.
	\item \textbf{Barok}:

	      Twórczość muzyków epoki baroku charakteryzowała się rozbudowaną ornamentacją.
	      Popularne stało się basso continuo, jako metoda prowadzenia akompaniamentu, oraz technika kontrapunktu,
	      w której kontrastujące melodie tworzyły złożone struktury dźwiękowe.
	      Kontynuowano również eksperymenty z harmonią i dynamiką.
	\item \textbf{Klasycyzm}:

	      W okresie klasycyzmu muzyczny proces twórczy cechowała kontrastująca z barokiem precyzja i klarowność.
	      Zaczęto stosować ścisłe formy — jak forma sonatowa — określające strukturę, tempo,
	      charakter i układ poszczególnych części utworów.
	      Wykorzystywane były powtarzalne motywy melodyczne; dążono do równowagi między poszczególnymi elementami składowymi utworów.
	\item \textbf{Romantyzm}:

	      W podobnym kontraście epoka romantyzmu przynosi duże oswobodzenie wyrazowe muzyki.
	      Kluczowym elementem motywującym twórczość staje się introspekcja, stąd utwory tego okresu nasycone są indywidualizmem
	      i wyrazistością emocjonalną.
	      Wciąż stosowane są pewne wzorce — gatunki, jak sonata czy fantazja, ale nie są one już tak ściśle określone,
	      stąd pojawiają się zróżnicowane eksperymenty kompozytorskie z nietypową harmonią, dynamicznymi kontrastami i rozbudowanymi
	      frazami melodycznymi.
	\item \textbf{XX wiek}:

	      W XX wieku obserwowane jest szczególne bogactwo i różnorodność podejść do kompozycji.
	      Pojawiają się tu zróżnicowane nurty, między innymi — impresjonizmu, ekspresjonizmu, neoklasycyzmu.
	      Muzyka staje się \enquote{produktem} licznych koncepcji i metod tworzenia — jak aleatoryzm, opierający
	      się o losowość lub serializm, kierowany przetworzeniami matematycznymi; a także idei — tu możnaby przytoczyć minimalizm.
	      Epoka ta przynosi jeszcze jeden przełom — komputery i ich zastosowanie w muzyce, jako narzędzia twórcze, nośniki i instrumenty.
\end{itemize}

\subsubsection{Notacja muzyczna}
Na przestrzeni epok kluczowym nośnikiem pozwalającym na utrwalenie melodii była notacja.
Ona również ewoluowała i zmieniała się, dostosowując się do zaawansowania rozwoju muzyki oraz potrzeb artystów.
Zauważalne jest tu zróżnicowanie konceptualne zapisu melodii.
\begin{itemize}
	\item \textbf{Starożytność}:

	      Najstarsza forma notacji muzycznej przypisywana jest Babilończykom (1400 rok p.n.e.), opiera się o zapis klinowy.
	      Podobnie późniejsze odkrycia — pochodzące ze Starożytnej Grecji — również opierają się o alfabet,
	      a także symbole określające długość dźwięku.
	\item \textbf{Średniowiecze}:

	      Pojawia się wówczas notacja dedykowana stricte muzyce — neumy. Pierwsza notacja tego typu wywodzi się z Imperium Bizantyjskiego,
	      jednak pod tym pojęciem rozumie się przede wszystkim europejską notację wykorzystywaną początkowo w chorale
	      gregoriańskim — do określania wysokości dźwięków oraz ich czasu trwania.
	      Epoka ta przynosi również specyficzny rodzaj nazewnictwa dźwięków — solmizację.
	\item \textbf{Renesans}:

	      Powstają podwaliny współczesnej, ogólnej notacji nutowej: zaczynają być stosowane nuty oraz symbole
	      — jak na przykład klucze.
	      Na przestrzeni kolejnych epok zapis jest rozwijany, umożliwiając bardzo dokładne określanie zamierzonego przez twórcę przebiegu
	      melodycznego.
	      Jednocześnie pojawiają się tabulatury — zapis ukierunkowany pod konkretny instrument, dostosowany pod jego charakterystykę —
	      przykładowo organy, lutnie itd.
	\item \textbf{Barok}:

	      Zaczyna być stosowany bas cyfrowany.
	      Jest to notacja numeryczna określające harmonię akompaniamentu utworów.
	      Rola ta przypadała zwykle konkretnym instrumentom — jak na przykład klawesynowi.
	      Zasada działania tego rodzaju notacji opierała się o akordy i relacje interwałowe pomiędzy poszczególnymi dźwiękami.
	\item \textbf{Klasycyzm i romantyzm}:

	      Epoki przynoszą dalszy rozwój standaryzowanego zapisu nutowego, celem oznaczenia ścisłej artykulacji, dynamiki i agogiki
	      dla wykonawców utworów.
	\item \textbf{XX wiek}:

	      Muzycy zaczynają eksperymentować z zapisem muzycznym, zupełnie dostosowując go do swoich potrzeb, tworząc warstwy abstrakcji,
	      własne \enquote{paradygmaty}, odwzorowujące eksperymentalność tworzonej przez nich muzyki.
	\item \textbf{Współczesność}:

	      Współcześnie korzysta się z wielu form zapisu muzycznego. Kluczową rolę odgrywa zapis nutowy,
	      lecz niektórzy artyści wciąż czasami sięgają po własne formy notacji, podobnie jak miało to miejsce w XX wieku.
	      Wielką popularnością cieszą się również tabulatury, zapis akordowy oraz wizualizacje komputerowe, z uwagi ta ich przystępność.
	      Jednocześnie w pewnych okolicznościach wciąż wykorzystywane są starsze metody zapisu — sięgające nawet do neum.
\end{itemize}

\subsubsection{Podsumowanie}
Jak więc widać, współcześnie wykorzystuje się szeroką gamę technik i sposobów notacji muzycznej. Dostępność edukacji muzycznej
oraz instrumentów przekłada się na duże zróżnicowanie \enquote{warsztatów} artystów. W następstwie XX—wiecznych tendencji kompozytorskich
oraz przemian społeczno—kulturowych osiągnięta zostaje całkowita dowolność w sposobach pracy z muzyką.

\subsection{Badanie rynku oprogramowania wspierającego pracę muzyków}
\subsubsection{Programy komputerowe}
Praca twórcza muzyków od XX wieku jest również wspierana przez programy komputerowe. Można zaobserwować dynamikę postępu
w tej dziedzinie, analizując następujące po sobie nowatorskie rozwiązania programistyczne na przestrzeni lat:
\begin{itemize}
	\item \textbf{1951 - początki zastosowania komputerów w muzyce}:

	      W 1951 roku inżynier komputerowy Max Mathews stworzył pierwszy program komputerowy do generowania dźwięku.
	      Program ten działał na komputerze IBM 704 i otworzył drzwi do eksperymentów z syntezą dźwięku przy użyciu komputerów.
	      W późniejszych latach powstaje "MUSIC I" — tego samego twórcy — umożliwiający tworzenie muzyki.
	      Można określić ten moment jako początek muzyki komputerowej.
	\item \textbf{1980 - MIDI (Musical Instrument Digital Interface) i edycja dźwięku}:

	      Wprowadzenie standardu MIDI umożliwiło komunikację między różnymi instrumentami muzycznymi a komputerami,
	      co znacząco ułatwiło produkcję i edycję muzyki przy użyciu oprogramowania.
	      W latach 80' pojawiły się też programy takie jak SoundEdit i Sound Forge, umożliwiające edycję dźwięku.
	      Zyskały one popularność w dziedzinie produkcji muzycznej.
	\item \textbf{1990 - DAW (Digital Audio Workstation), syntezatory i samplowanie}:

	      Powstanie programów DAW, takich jak Pro Tools, Cubase, Logic Pro, Ableton Live i FL Studio,
	      umożliwiło profesjonalistom tworzenie, edycję i produkcję muzyki na komputerach osobistych.
	      DAW stały się kluczowym narzędziem w przemyśle muzycznym.
	      Pojawiło się również oprogramowanie, które umożliwia emulację syntezatorów analogowych oraz samplowanie dźwięków.
	\item \textbf{2000 — Rozszerzanie zastosowań programów komputerowych}:

	      Oprogramowanie takie jak Kontakt, Omnisphere czy Massive pozwoliło na tworzenie wirtualnych instrumentów i efektów
	      dźwiękowych, co znacząco rozszerzyło możliwości produkcji muzycznej.
	      Powstają programy automatyzujące procesy: Ludwig 3.0 — tworzący automatyczne aranżacje, Transcribe — transkrybujący
	      dźwięk na notację muzyczną, jak również cała gama mniejszych i większych projektów wspomagających muzyków w procesie
	      twórczym.
\end{itemize}

\subsubsection{Aplikacje mobilne}
Współcześnie, przy wzroście możliwości smartfonów należałoby oczekiwać również wzrostu ich zastosowań w dziedzinie
muzyki. Rynek jednak oferuje dość nieliczne takie rozwiązania — często nastawione na wykonywanie pojedycznych,
prostych czynności. Za przykład może posłużyć tu Maestro — umożliwiający tworzenie zapisu nutowego lub aplikacje
służące za metronomy.

Znaczący potencjał aplikacji mobilnych jest jednak widoczny na przykładzie GarageBand. Jest to w zasadzie pełne DAW,
pozwalające na tworzenie muzyki przy użyciu smartfona. Umożliwia korzystanie z syntezatorów, nagrywanie, edycję dźwięku,
jak również możliwość nakładania efektów dźwiękowych. Mimo to, w przypadku tego rodzaju pracy, preferowane są
z rozwiązania desktopowe, które — choćby z uwagi na rozmiar ekranu — oferują większy komfort.

\subsubsection{Podsumowanie}
Analiza programów i aplikacji stanowiących zestaw narzędzi muzyków na przestrzeni lat oraz współcześnie
odsłania niezapełnioną niszę: wciąż brakuje rozwiązań — tak wśród aplikacji mobilnych, jak i programów komputerowych —
które mogłoby wspomóc pracę twórczą w fazach powstawania koncepcji, podczas aranżacji utworów muzycznych. Dotychczasowe
rozwiązania pozwalają zapisywać ułożone utwory oraz poddawać je edycji; także umożliwiają generowanie aranżacji bądź całych
utworów. Jednak elementy procesu aranżacji dokonywanej przez muzyków, wiążące się z mnogością pomysłów i licznych zmian —
nieczęsto są wspierane przez rozwiązania programistyczne.

\subsection{Rozwiązania oferowane przez inne aplikacje niepowiązane z procesem aranżacji}
Aby odpowiedzieć na poruszony wyżej problem, konieczna jest analiza oprogramowania przynoszącego rozwiązania problemów
o podobnym charakterze. Najbliższym przykładem będzie tu oprogramowanie umożliwiające tworzenie notatek.
Na poniższych przykładach, można zaobserwować rozwój przełomowych — a współcześnie kluczowych — funkcjonalności oferowanych
przez aplikacje z tej kategorii:
\begin{itemize}
	\item \textbf{1987 - Lotus Agenda}:

	      Lotus Agenda, stworzony przez firmę Lotus Development Corporation oparty, był jednym z pierwszych programów do
	      zarządzania notatkami.
	      Umożliwiał użytkownikom tworzenie notatek, organizowanie ich w kategorie i przeszukiwanie.
	\item \textbf{2000 - EverNote}:

	      Evernote, założone przez Stepana Pachikova, stało się jednym z najbardziej rozpoznawalnych narzędzi do tworzenia
	      notatek.
	      Aplikacja pozwala na tworzenie notatek w różnych formatach, w tym tekst, obrazy, dźwięki,
	      oraz synchronizację ich między różnymi urządzeniami.
	\item \textbf{2003 - Microsoft OneNote}:

	      Microsoft OneNote (pierwornie OneNote 2003), zyskał popularność jako narzędzie do tworzenia notatek w systemie
	      Windows.
	      W kolejnych edycjach umożliwiał użytkownikom innowacyjne organizowanie notatek w ramach \enquote{tablicy}.
	\item \textbf{2008 - Evernote jako aplikacja mobilna}:

	      Zmiana kierunku rozwoju aplikacji, wejście na rynek aplikacji mobilnych doprowadziła do ogromnego wzrostu popularności Evernote,
	      co ukazało duży potencjał aplikacji do notowania, będących zawsze \enquote{pod ręką} użytkowników.
	      Inne aplikacje również podążają za tym trendem.
	\item \textbf{2013 - Notion}:

	      Znaczącą innowację przynosi Notion, oferując dotąd niespotykany wachlarz możliwości.
	      Stanowi kolejny krok w ewolucji aplikacji do notowania — jako tzw. personalna baza wiedzy (eng.
	      \enquote{Personal Knowledge Base})
	      O ile sama aplikacja charakteryzuje się stosunkową prostotą, o tyle — jest to swoista \enquote{piaskownica danych},
	      dowolnie organizowanych i personalizowanych przez użytkownika, przy użyciu notatników, kalendarzy i tabel.
	\item \textbf{2020 - Obsidian, Logseq}:

	      Kolejnym przełomowym krokiem rozwoju jest Obsidian. Charakteryzuje się większą prostotą od Notion,
	      gdyż działa w oparciu o pliki markdown, jednak użytkownik może go dowolnie dostosować do swych potrzeb m.in.
	      za sprawą
	      wtyczek. W ślad za nim podąża Logseq, oprogramowanie Open-Source, które opiera się grafy i listy punktowane.
\end{itemize}
Na przykładzie najnowszych aplikacji, widoczna jest tendencja, oddawania w ręce użytkowników prostych narzędzi,
które mogą być w możliwie szeroki sposób wykorzystane przez użytkownika — zgodnie z jego zamysłem i stylem pracy.
Należałoby w tym miejscu odnotować także kluczowe cechy współczesnych aplikacji do notowania: mobilność, elastyczność
i możliwość organizacji.
Wymienione aplikacje reprezentują także różne podejścia do wizualizacji \enquote{notatki}.
Może być to strona z tekstem — Obsidian, element na liście punktowanej — Logseq, karta w układzie kanban — Notion,
element umiejscowiony (w dowolnym położeniu na) tablicy — Microsoft OneNote itd. Warto tu jednak zwrócić uwagę,
że znacząca część współczesnych rozwiązań umożliwia również wizualną || przestrzenną organizację notatek.
