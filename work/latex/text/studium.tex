\subsection{Definicja problematyki procesu aranżacji}
Aranżacją w ujęciu muzycznym jest
\enquote{opracowanie utworu muzycznego w nowym stylu bez zasadniczych zmian linii melodycznej} \cite{wsjp}.
Stoi on z tego względu konceptualnie pomiędzy procesami kompozycji oraz interpretacji utworu
i w efekcie współdzieli częściowo problematykę obu tych etapów, jako proces techniczny oraz twórczy.
Jakość aranżacji zależy od znajomości teorii muzyki, technik kompozytorskich, jak również doświadczenia w pracy z muzyką.
Podczas tego procesu, muzyk nadaje utworowi swoisty kształt. Jest to efekt dynamicznej i zmiennej pracy kreatywnej, podczas
której wyłania się wiele muzycznych koncepcji, wymagających uporządkowania i selekcji \cite{orchestration}.

\subsection{Rys historyczny muzyki klasycznej}
Analiza historii muzyki na przestrzeni epok odsłania charakter procesu aranżacji.
W kontekście omawianego zagadnienia szcególnie istotna jest ewolucja technik twórczych, jak również samej
notacji muzycznej. Ukazują one zróżnicowanie podejść twórczych i sposobów pracy z muzyką.

\subsubsection{Rozwój technik procesu twórczego muzyki}
Rozwój technik wykorzystywanych przez kompozytorów i aranżerów tworzy krajobraz muzyczny następujących po sobie epok.
Widoczne jest zróżnicowanie popularnych tendencji wśród twórców muzyki w Europie.
\begin{itemize}
	\item \textbf{Średniowiecze}:

	      Proces twórczy muzyków epoki średniowiecza jest głęboko osadzony w liturgicznym kontekście religijnym.
	      Znaczącą rolę ogdrywa chorał gregoriański jako podstawa tworzonych utworów.
	      Kompozytorzy tego okresu opierają się na zawartych w nim melodiach, wykorzystując techniki — jak na przykład organum,
	      melizmaty oraz schematy rytmiczne — urozmajcając ich brzmienie.
	      Organum polega na dodawaniu nowych dźwięków do melodii chorału, celem tworzenia wielowarstwowych melodii;
	      melizmaty stosowa są tu jako muzyczne \enquote{ozdobniki}. Kluczową rolę w utworze odgrywa jego tekst \cite{abc}.
	\item \textbf{Renesans}:

	      Odkryciem przełomowym epoki renesansu jest polifonia — nakładanie na siebie równoważnych głosów o
	      odmiennych liniach melodycznych.
	      Szczególną popularnością cieszy się technika imitacji — naśladownictwa.
	      Stosowane są bardziej skomplikowane rytmy, na przykład — opierające się na trójdzielności — oraz nowe
	      skale muzyczne \cite{atlas1}.
	\item \textbf{Barok}:

	      Twórczość muzyków epoki baroku charakteryzuje się rozbudowaną ornamentacją.
	      Popularne staje się basso continuo — jako metoda prowadzenia akompaniamentu — oraz technika kontrapunktu,
	      w której kontrastujące melodie tworzą złożone struktury dźwiękowe.
	      Kontynuowane są również eksperymenty z harmonią i dynamiką utworów \cite{estetyka}.
	\item \textbf{Klasycyzm}:

	      W okresie klasycyzmu muzyczny proces twórczy cechuje kontrastująca z barokiem precyzja i klarowność.
	      Stosowane są ścisłe formy — jak na przykład forma sonatowa — określające strukturę, tempo,
	      charakter i układ poszczególnych części utworów.
	      Wykorzystywane są powtarzalne motywy melodyczne;
	      widoczne jest dążenie do równowagi między poszczególnymi elementami składowymi utworów \cite{abc}.
	\item \textbf{Romantyzm}:

	      W podobnym do epok wcześniejszych kontraście epoka romantyzmu przynosi duże oswobodzenie wyrazowe muzyki.
	      Kluczowym elementem motywującym twórczość jest introspekcja, stąd utwory tego okresu nasycone są indywidualizmem
	      i wyrazistością emocjonalną.
	      Wciąż stosowane są pewne gatunki stanowiące \enquote{wzorce}, jak sonata czy fantazja,
	      lecz nie są one już tak ściśle określone, jak w klasycyzmie.
	      Pojawiają się stąd zróżnicowane eksperymenty kompozytorskie z nietypową harmonią, dynamicznymi kontrastami i
	      rozbudowanymi frazami melodycznymi \cite{estetyka}.
	\item \textbf{XX wiek}:

	      W XX wieku obserwowane jest szczególne bogactwo i różnorodność podejść do kompozycji.
	      Widoczne jest zróżnicowanie nurtów — między innymi impresjonizmu, ekspresjonizmu, neoklasycyzmu.
	      Muzyka staje się \enquote{produktem} licznych koncepcji i metod tworzenia — jak aleatoryzm, opierający
	      się o losowość lub serializm, kierowany przetworzeniami matematycznymi; a także idei — można tu przytoczyć minimalizm.
	      Epoka ta przynosi jeszcze jeden przełom — komputery i ich zastosowanie w muzyce,
	      jako narzędzia twórcze, instrumenty oraz nośniki danych \cite{atlas2}.
\end{itemize}

\subsubsection{Notacja muzyczna}
Na przestrzeni epok kluczowym nośnikiem pozwalającym na utrwalenie melodii jest notacja.
Ona również zmienia się i ewoluuje, dostosowując się do rozwoju muzyki oraz potrzeb artystów.
Na przestrzeni epok zauważalne jest stąd zróżnicowanie konceptualne zapisu melodii.
\begin{itemize}
	\item \textbf{Starożytność}:

	      Najstarsza forma notacji muzycznej — oparta o zapis klinowy — pochodzi z Mezopotamii (1400 rok p.n.e.).
	      Podobnie późniejsze odkrycia na terenie Grecji — również opierają się o alfabet
	      oraz symbole określające długość dźwięku \cite{atlas1}.
	\item \textbf{Średniowiecze}:

	      Pojawia się notacja dedykowana ściśle muzyce — neumy. Określenie to odnosi się do kilku rodzajów notacji, z których pierwsza wywodzi
	      się z Imperium Bizantyjskiego, jednak pod tym pojęciem rozumie się przede wszystkim europejską notację wykorzystywaną początkowo w chorale
	      gregoriańskim — do określania wysokości dźwięków oraz ich czasu trwania.
	      Epoka średniowiecza przynosi również specyficzny rodzaj nazewnictwa dźwięków — solmizację \cite{abc}.
	\item \textbf{Renesans}:

	      Powstają podwaliny współczesnej ogólnej notacji nutowej: zaczynają być stosowane nuty oraz symbole
	      — jak na przykład klucze.
	      Na przestrzeni kolejnych epok zapis jest rozwijany, umożliwiając bardzo dokładne określanie zamierzonego przez twórcę przebiegu
	      melodycznego.
	      Jednocześnie pojawiają się tabulatury — zapis ukierunkowany pod konkretny instrument, dostosowany do jego charakterystyki —
	      dla lutni, organów itp. \cite{atlas1}.
	\item \textbf{Barok}:

	      Zaczyna być stosowany bas cyfrowany.
	      Jest to notacja numeryczna określające harmonię akompaniamentu utworów.
	      Rola ta przypada zwykle konkretnym instrumentom, najczęściej — klawesynowi.
	      Zasada działania tego rodzaju notacji opiera się o akordy i relacje interwałowe pomiędzy poszczególnymi dźwiękami.
	\item \textbf{Klasycyzm i romantyzm}:

	      Kolejne epoki przynoszą dalszy rozwój standaryzowanego zapisu nutowego, celem oznaczenia ścisłej artykulacji, dynamiki i agogiki
	      dla wykonawców utworów \cite{abc}.
	\item \textbf{XX wiek}:

	      Muzycy zaczynają eksperymentować z zapisem muzycznym, zupełnie dostosowując go do osobistych potrzeb, tworząc warstwy abstrakcji,
	      własne \enquote{paradygmaty}, odwzorowujące innowacyjność tworzonej muzyki \cite{atlas2}.
	\item \textbf{Współczesność}:

	      Współcześnie korzysta się z wielu form zapisu muzycznego. Kluczową rolę odgrywa zapis nutowy,
	      lecz niektórzy artyści wykorzystują także inne formy notacji, podobnie jak miało to miejsce w XX wieku.
	      Jednocześnie w pewnych okolicznościach wciąż wykorzystywane są starsze metody zapisu — sięgające nawet do neum.
	      Wielką popularnością cieszą się również tabulatury, zapis akordowy oraz wizualizacje komputerowe — z uwagi ta ich przystępność \cite{modern}.
\end{itemize}

\subsubsection{Podsumowanie}
Analiza rysu historycznego ukazuje szeroką gamę wykorzystywanych technik i sposobów notacji muzycznej. Dostępność edukacji muzycznej
oraz instrumentów przekłada się na duże zróżnicowanie sposobów pracy artystów. W następstwie XX—wiecznych tendencji kompozytorskich
oraz przemian społeczno—kulturowych osiągnięta zostaje całkowita dowolność w sposobach pracy z muzyką.

\subsection{Badanie rynku oprogramowania wspierającego pracę muzyków}
\subsubsection{Programy komputerowe}
Praca twórcza muzyków od XX wieku jest również wspierana przez programy komputerowe. Można zaobserwować dynamikę postępu
w tej dziedzinie, na przykładzach następujących po sobie nowatorskich rozwiązań programistycznych na przestrzeni lat:
\begin{itemize}
	\item \textbf{1950 - początki zastosowania komputerów w muzyce}:

	      W 1951 roku inżynier komputerowy Max Mathews tworzy pierwszy program komputerowy do generowania dźwięku.
	      Program, działając na komputerze IBM 704, otwiera drzwi do eksperymentów z syntezą dźwięku przy użyciu komputerów.
	      W późniejszych latach powstaje "MUSIC I" — tego samego twórcy — umożliwiający tworzenie muzyki.
	      Można określić ten moment jako początek muzyki komputerowej \cite{50}.
	\item \textbf{1980 - MIDI (Musical Instrument Digital Interface) i edycja dźwięku}:

	      Wprowadzenie standardu MIDI umożliwia komunikację między różnymi instrumentami muzycznymi a komputerami,
	      co znacząco ułatwia produkcję i edycję muzyki przy użyciu oprogramowania.
	      Pojawiają się też programy takie jak SoundEdit i Sound Forge, umożliwiające edycję dźwięku.
	      Zyskują one popularność w dziedzinie produkcji muzycznej \cite{90}.
	\item \textbf{1990 - DAW (Digital Audio Workstation), syntezatory i samplowanie}:

	      Powstanie programów DAW, takich jak Pro Tools, Cubase, Logic Pro, Ableton Live i FL Studio,
	      umożliwia profesjonalistom tworzenie, edycję i produkcję muzyki na komputerach osobistych.
	      DAW stają się kluczowym narzędziem w przemyśle muzycznym.
	      Pojawia się również oprogramowanie umożliwiające emulację syntezatorów analogowych oraz samplowanie dźwięków \cite{90}.
	\item \textbf{2000 — Rozszerzanie zastosowań programów komputerowych}:

	      Oprogramowanie takie jak Kontakt, Omnisphere czy Massive pozwala na tworzenie wirtualnych instrumentów i efektów
	      dźwiękowych, rozszerzając możliwości produkcji muzycznej.
	      Powstają programy automatyzujące procesy jak Ludwig 3.0 — tworzący automatyczne aranżacje, Transcribe — transkrybujący
	      dźwięk na notację muzyczną, a także szereg innych projektów wspomagających muzyków w procesie twórczym \cite{00}.
\end{itemize}

\subsubsection{Aplikacje mobilne}
Współcześnie, przy wzroście możliwości smartfonów należałoby oczekiwać również wzrostu ich zastosowań w dziedzinie
muzyki. Rynek jednak oferuje dość nieliczne rozwiązania — często nastawione na wykonywanie pojedycznych,
prostych czynności — istnieją grupy aplikacji umożliwiających tworzenie zapisu nutowego lub służących za metronomy \cite{10}.
Znaczący potencjał aplikacji mobilnych jest za to widoczny na przykładzie GarageBand, będącego pełnym programem DAW,
pozwalającym na tworzenie muzyki przy użyciu smartfona. Umożliwia korzystanie z syntezatorów, nagrywanie, edycję dźwięku,
jak również możliwość nakładania efektów dźwiękowych. Mimo to, w przypadku tego rodzaju pracy, preferowane są
z rozwiązania desktopowe, które — choćby z uwagi na rozmiar ekranu — oferują większy komfort \cite{interaction}.

\subsubsection{Podsumowanie}
Analiza programów i aplikacji stanowiących zestaw narzędzi muzyków na przestrzeni lat oraz współcześnie
odsłania niezapełnioną niszę: wciąż brakuje rozwiązań — tak wśród aplikacji mobilnych, jak i programów komputerowych —
które mogłoby wspomóc pracę twórczą w fazach powstawania koncepcji, podczas aranżacji utworów muzycznych. Dotychczasowe
rozwiązania pozwalają zapisywać ułożone utwory oraz poddawać je edycji; także umożliwiają generowanie aranżacji bądź całych
utworów. Jednocześnie elementy procesu aranżacji muzycznej, wiążące się z mnogością pomysłów i zmian —
nieczęsto są wspierane przez rozwiązania programistyczne.

\subsection{Rozwiązania oferowane przez inne aplikacje niepowiązane z procesem aranżacji}
Aby odpowiedzieć na problem omawiany w pracy, konieczna jest analiza oprogramowania przynoszącego rozwiązania problemów
o podobnym charakterze. Pod tym kątem, najbliższym przykładem jest oprogramowanie umożliwiające
tworzenie notatek. Można zaobserwować rozwój wielu przełomowych — a współcześnie kluczowych — funkcjonalności oferowanych
przez aplikacje z tej kategorii na przestrzeni ostatnich dwudziestu lat:
\begin{itemize}
	\item \textbf{2000 - EverNote}:

	      Evernote, założone przez Stepana Pachikova, jest jednym z najbardziej rozpoznawalnych narzędzi do tworzenia
	      notatek. Aplikacja pozwala tworzyć notatki w różnych formatach, jako tekst, obrazy, dźwięki,
	      oraz umożliwia synchronizację ich między różnymi urządzeniami \cite{evernote}.
	\item \textbf{2003 - Microsoft OneNote}:

	      Microsoft OneNote (pierwornie OneNote 2003), jest początkowo narzędziem do tworzenia notatek w systemie
	      Windows. W kolejnych edycjach umożliwia użytkownikom innowacyjne organizowanie notatek w ramach \enquote{tablicy} \cite{onenote}.
	\item \textbf{2008 - Evernote jako aplikacja mobilna}:

	      Zmiana kierunku rozwoju aplikacji za sprawą wejścia na rynek aplikacji mobilnych doprowadza do wzrostu
	      popularności Evernote,
	      co ukazuje potencjał aplikacji do notowania, będących zawsze \enquote{pod ręką} użytkowników.
	      Inne aplikacje również podążają za tym trendem \cite{evernote}.
	\item \textbf{2013 - Notion}:

	      Znaczącą innowację przynosi Notion. Stanowi kolejny krok w ewolucji aplikacji do notowania, jako tzw. personalna
	      baza wiedzy (eng. \enquote{Personal Knowledge Base}).
	      Choć sama aplikacja charakteryzuje się stosunkową prostotą, jest jednocześnie to swoistą \enquote{piaskownicą danych},
	      dowolnie organizowanych przez użytkownika, przy użyciu notatników, kalendarzy i tabel \cite{notion}.
	\item \textbf{2020 - Obsidian, Logseq}:

	      Kolejny przełomowy kierunek rozwoju wyznacza aplikacja Obsidian. Charakteryzuje się większą prostotą od Notion,
	      działając w oparciu o pliki markdown, jednak użytkownik może ją dowolnie dostosować do swych potrzeb m.in.
	      za sprawą wtyczek \cite{obsidian}.
	      Podobne założenia realizowane są przez Logseq,
	      oprogramowanie open-source, które opiera się o listy punktowane oraz grafy \cite{logseq}.
\end{itemize}
Na przykładzie najnowszych aplikacji, widoczna jest tendencja oddawania w ręce użytkowników prostych narzędzi,
które mogą być wykorzystane na wiele sposobów — zgodnie z jego zamysłem i stylem pracy.
Wyłaniają się stąd również kluczowe założenia współczesnych aplikacji do notowania: mobilność, elastyczność i organizacja.
Wymienione aplikacje reprezentują także różne podejścia do wizualizacji \textit{notatki}.
Może być to strona z tekstem (Obsidian), element na liście punktowanej (Logseq), karta w układzie kanban (Notion),
element umiejscowiony w dowolnym położeniu na tablicy (Microsoft OneNote) itd.
Współczesne trendy wśród aplikacji do notowania odsłaniają również szczególne znaczenie wizualnej || przestrzennej organizacji notatek.