Celem pracy inżynierskiej zaprojektowanie i zaimplementowanie aplikacji do zarządzania i obsługi wydarzeń uczelnianych. Cel ten został w pełni spełniony, uwzględniając wszystkie wymagania funkcjonalne i niefunkcjonalne, które zostały postawione przed systemem.

Motywacją do jej stworzenia był fakt istnienia podobnych rozwiązań, które były płatne, bądź przystosowane do dużych wydarzeń cyklicznych jak np. koncerty czy sztuki teatralne. W przypadku wydarzeń uczelnianych, które są organizowane przez koła naukowe czy studentów często nie ma możliwości na wykupienie takich rozwiązań. Dlatego też powstała aplikacja, która jest darmowa i dostępna dla każdego, kto chciałby z niej skorzystać.
\subsection {Efekt końcowy pracy}
Rezultatem wielomiesięcznej pracy jest wytworzona aplikacja. Pozwala ona na administrację wydarzeniem, zarządzanie uczestnikami, a także generowanie kodów QR. Użytkownicy mogą przeglądać nagrody, wydarzenia i się na nie zapisywać. Dodatkowo mogą skanować kody QR, które wynagradzają ich punktami. Punkty następnie  są traktowane jako losy w module losowania nagród.

Implementacja systemu pozwoliła autorowi pracy na zdobycie nowych umiejętności związanych z wytwarzaniem oprogramowania oraz poznanie nowych technologii. Pozwoliło to na rozwój umiejętności programistycznych, które będą przydatne w karierze zawodowej.

Autor pracy napotkał na kilka problemów podczas wytwarzania aplikacji. Największym z nich była współpraca z aparatem w ramach aplikacji mobilnej, w celu skanowania kodów QR. Wymagało to wielu prób i błędów, aby znaleźć rozwiązanie, które działało poprawnie. Autor nie posiadał wiedzy na temat tworzenia aplikacji mobilnej, więc wymagało to wielu godzin czytania dokumentacji.
\subsection {Możliwe kierunki rozwoju}
Projekt aplikacji posiada wiele możliwości rozwoju o funkcjonalności, które znacznie wykraczały poza zakres czasowy przeznaczony na pracę inżynierską. Większość z nich zostanie zaimplementowana przez autora pracy w ramach własnego rozwoju systemu.

Pierwszą z nich jest implementacja roli "Prowadzącego". Zapewnienie mu możliwości kontaktu z uczestnikami jego warsztatów w celu przesłania materiałów, które zostały wykorzystane podczas zajęć. Dodatkowo możliwość integracji interaktywnych quizów, które mogłyby być rozwiązane przez uczestników w czasie trwania wydarzenia, co premiowałoby dodatkowymi punktami, bądź osobnymi nagrodami - losowanymi przez prowadzącego.

Kolejną funkcjonalnością, która mogłaby zostać zaimplementowana jest interaktywna mapa. Z wykorzystaniem modułu GPS, użytkownik mógłby zobaczyć, gdzie znajduje się w stosunku do miejsca warsztatów, na które się zapisał. Następnie dotrzeć do celu, korzystając z instrukcji wyświetlanych na ekranie telefonu.

Przydatnym modułem dla systemu byłby system statystyk. Umożliwiłby on zbieranie danych i ich analizę. Dzięki temu organizatorzy mogliby sprawdzić, które wydarzenia cieszyły się największą popularnością, a które nie. Dodatkowo mogliby sprawdzić, które z nich były najczęściej odwiedzane przez użytkowników, a które nie. Dzięki temu mogliby w przyszłości lepiej planować wydarzenia, które organizują a co za tym idzie zwiększyć ilość uczestników.

Dużym potencjałem rozwoju jest również wykorzystanie kodów QR i ich skanowania w większym zakresie. Możliwe byłoby użycie ich do interakcji ze stoiskami partnerskimi i ich promocją. Rozbudowa systemu generowania kodu QR o np. umieszczanie loga partnerów w kodzie QR mogłaby przyczynić się do zwiększenia zainteresowania danym stoiskiem.

Czasem zdarza się, że wydarzenie jest odwoływane, przeniesione bądź zmodyfikowane z przyczyn niezależnych od organizatorów. W takim przypadku przydatnym modułem byłby system powiadomień. Użytkownicy, którzy zapisali się na dane wydarzenie otrzymaliby powiadomienie na swoim telefonie o zmianach, które zostały wprowadzone. Dodatkowo mogliby także otrzymywać różne informacje, które byłyby wysyłane przez organizatorów, czy przypomnienia o zbliżającym się wydarzeniu.