React - biblioteka JavaScript do tworzenia interaktywnych interfejsów użytkownika. Fundamentalnym założeniem są komponenty, reprezentujące modularne i hermetyczne jednostki interfejsu. Komponenty te umożliwiają organizację kodu w sposób przejrzysty i łatwy do zarządzania, co przyczynia się do skalowalności projektów. \autocite{react}

Centralną koncepcją w React jest reaktywność, co oznacza, że interfejs użytkownika jest automatycznie aktualizowany w odpowiedzi na zmiany w stanie danych aplikacji. Framework wykorzystuje wirtualny DOM (Document Object Model), co przyczynia się do efektywnej aktualizacji tylko tych fragmentów interfejsu, które uległy zmianie, zamiast całej struktury. To podejście minimalizuje obciążenie i przyspiesza renderowanie widoków.

Biblioteka używa składni JSX, co pozwala na zapisywanie struktury interfejsu przy użyciu składni przypominającej HTML. JSX integruje się z JavaScriptem, umożliwiając programistom tworzenie bardziej czytelnych i zwięzłych opisów interfejsu. React oferuje również narzędzia deweloperskie, które wspierają proces debugowania i optymalizacji kodu.