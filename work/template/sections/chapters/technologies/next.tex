Next.js to popularny framework Javascript do tworzenia aplikacji internetowych o otwartym kodzie źródłowym. Został stworzony przez firmę Vercel w 2016 roku. Jest to narzędzie oparte na bibliotece React, służącej do tworzenia interaktywnych interfejsów użytkownika.

Rozbudowuje on możliwości React o kilka kluczowych funkcji, które pozwalają na tworzenie aplikacji internetowych o złożonej architekturze. Umożliwia on renderowanie aplikacji po stronie serwera, co przyspiesza ładowanie strony oraz poprawia jej pozycjonowanie w wyszukiwarkach. Z użyciem Next.js możemy także generować statyczne strony internetowe. Framework oferuje także wiele wbudowanych funkcji, które pozwalają na łatwe zarządzanie metadanymi, obsługę dynamicznych ścieżek oraz czy plików statycznych. \autocite{nextjs}