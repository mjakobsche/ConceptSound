W niniejszej pracy zastosowano technologie, które spełniają ściśle określone kryteria, zgodnie z zainteresowaniami autora. Wybrane technologie zostały ocenione pod kątem poniższych parametrów:

\begin{itemize}
    \item \textbf{Otwarty kod źródłowy:} Wybrane technologie opierają się na otwartym kodzie źródłowym, co umożliwia pełen dostęp, analizę, oraz ewentualne dostosowanie do indywidualnych potrzeb.
    
    \item \textbf{Darmowe:} Wszystkie technologie są dostępne bez dodatkowych opłat, co jest istotne z perspektywy ekonomicznej.
    
    \item \textbf{Popularne:} Wybór technologii uwzględniał ich popularność w branży, co zapewnia szeroką dostępność dokumentacji oraz wsparcie społeczności.
    
    \item \textbf{Duża społeczność:} Wykorzystane technologie cieszą się uznaniem w społeczności, co gwarantuje aktywną wymianę wiedzy oraz dostępność rozwiązań problemów.
    
    \item \textbf{Ogólnodostępne:} Technologie są powszechnie dostępne dla użytkowników o różnym poziomie doświadczenia, co ułatwia ich adaptację.
    
    \item \textbf{Regularne aktualizacje:} Wybrane technologie są systematycznie aktualizowane, co zapewnia korzystanie z najnowszych funkcji oraz zabezpieczeń.
\end{itemize}

Takie cechy gwarantują aktywne wsparcie oraz dynamiczny rozwój zastosowanej technologii, co korzystnie wpływa na dalsze perspektywy rozwoju aplikacji oraz skuteczne rozwiązywanie napotkanych problemów napotkanych podczas procesu tworzenia aplikacji.

\subsection{React}
React - biblioteka JavaScript do tworzenia interaktywnych interfejsów użytkownika. Fundamentalnym założeniem są komponenty, reprezentujące modularne i hermetyczne jednostki interfejsu. Komponenty te umożliwiają organizację kodu w sposób przejrzysty i łatwy do zarządzania, co przyczynia się do skalowalności projektów. \autocite{react}

Centralną koncepcją w React jest reaktywność, co oznacza, że interfejs użytkownika jest automatycznie aktualizowany w odpowiedzi na zmiany w stanie danych aplikacji. Framework wykorzystuje wirtualny DOM (Document Object Model), co przyczynia się do efektywnej aktualizacji tylko tych fragmentów interfejsu, które uległy zmianie, zamiast całej struktury. To podejście minimalizuje obciążenie i przyspiesza renderowanie widoków.

Biblioteka używa składni JSX, co pozwala na zapisywanie struktury interfejsu przy użyciu składni przypominającej HTML. JSX integruje się z JavaScriptem, umożliwiając programistom tworzenie bardziej czytelnych i zwięzłych opisów interfejsu. React oferuje również narzędzia deweloperskie, które wspierają proces debugowania i optymalizacji kodu.
\subsection{Next.js}
Next.js to popularny framework Javascript do tworzenia aplikacji internetowych o otwartym kodzie źródłowym. Został stworzony przez firmę Vercel w 2016 roku. Jest to narzędzie oparte na bibliotece React, służącej do tworzenia interaktywnych interfejsów użytkownika.

Rozbudowuje on możliwości React o kilka kluczowych funkcji, które pozwalają na tworzenie aplikacji internetowych o złożonej architekturze. Umożliwia on renderowanie aplikacji po stronie serwera, co przyspiesza ładowanie strony oraz poprawia jej pozycjonowanie w wyszukiwarkach. Z użyciem Next.js możemy także generować statyczne strony internetowe. Framework oferuje także wiele wbudowanych funkcji, które pozwalają na łatwe zarządzanie metadanymi, obsługę dynamicznych ścieżek oraz czy plików statycznych. \autocite{nextjs}
\subsection{Typescript}
Typescript to nadzbiór języka JavaScript, które dodaje statyczne typowanie oraz nowe funkcje do języka. Jest to język programowania wysokiego poziomu, który kompilowany jest do języka JavaScript. Prężnie rozwijany przez  firmę Microsoft, a jego pierwsza wersja została wydana w 2012 roku. TypeScript jest wykorzystywany przez wiele dużych firm. Jest to rozwiązanie, które jest szczególnie odpowiednie dla aplikacji, które wymagają ścisłego typowania lub są rozbudowane i wymagają częstych zmian. Typowanie statyczne może ułatwić tworzenie i debugowanie kodu, ponieważ może pomóc w wykrywaniu błędów typów. \autocite{typescript}
\subsection{Tailwind}
Tailwind CSS to popularne narzędzie, które znacząco przyśpiesza i ułatwia projektowanie stron internetowych poprzez dostarczanie gotowych klas CSS. Nie oferuje on gotowych komponentów, jak inne popularne biblioteki, a jedynie klasy CSS, które można wykorzystać do tworzenia własnych komponentów. Pozwala to na większą elastyczność i kontrolę nad wyglądem aplikacji. Tailwind automatycznie usuwa nieużywany CSS podczas budowania aplikacji, co pozytywnie wpływa na wydajność aplikacji. \autocite{tailwind}
\subsection{Shadcn UI}
Shadcn to kolekcja reużywalnych komponentów, które są wykorzystywane poprzez skopiowanie kodu i wklejenie go wedle własnych potrzeb. Nie instalujemy ich w tradycyjny sposób jako zależności. Komponenty możemy w pełni dostosowywać do własnych potrzeb. Zbiór ten obsługuje także ciemny motyw a wszystkie komponenty możemy zainstalować w wygodny sposób używając komend. \autocite{shadcn}
\subsection{Ionic Framework}
Ionic Framework to otwarto-źródłowy zestaw  ponad 100 komponentów do budowania nowoczesnych i wydajnych aplikacji mobilnych i progresywnych. Przystosowany do obsługi gestów, bardzo efektywny i lekki. Posiada wbudowany tryb ciemny, co pozwala na łatwe dostosowanie aplikacji do preferencji użytkownika. Nie jest zależny od żadnego popularnego środowiska do tworzenia aplikacji internetowych, jednakże jest kompatybilny z takimi środowiskami jak Angular, React czy Vue. \autocite{ionic}
\subsection{Clerk}
Clerk to kompletne rozwiązanie służące do uwierzytelniania oraz zarządzania użytkownikami. Posiada darmowy, przystępny plan dla mniejszych aplikacji. Współpracuje z wieloma popularnymi językami programowania. Dostarcza gotowe komponenty, znacznie przyśpieszając pracę nad aplikacją, które można dostosowywać do własnych potrzeb. Umożliwia zarządzanie użytkownikami, ich uprawnieniami oraz sesjami. Zapewnia bezpieczne uwierzytelnianie, które jest zgodne z najnowszymi standardami. Clerk pozwala na użycie kilkunastu dostawców uwierzytelniania jak Google, Apple, czy Facebook. Dodatkowo pozwala na integrację z własną bazą danych oraz możliwość wcielenia się w danego użytkownika przez administratora. \autocite{clerk}
\subsection{PostgreSQL}
PostgreSQL jest relacyjną bazą danych typu open source, która jest rozwijana od 1986 roku. Jest to jeden z najbardziej popularnych systemów zarządzania bazą danych na świecie, wykorzystywany w szerokim zakresie zastosowań, w tym w aplikacjach internetowych czy systemach zarządzania treścią. System ten cechuje się wysoką stabilnością i niezawodnością. PostgreSQL oferuje szereg funkcjonalności jak obsługa transakcji, replikację czy pełen wyszukiwanie tekstowe. Dzięki dużej społeczności i otwartym kodzie źródłowym, PostgreSQL jest stale rozwijany i udoskonalany o nowe funkcjonalności czy kluczowe poprawki bezpieczeństwa. \autocite{postgres}
\subsection{Prisma}
Prisma ORM to zaawansowane narzędzie służące do mapowania obiektowo-relacyjnego (ORM) w kontekście aplikacji opartych na języku TypeScript lub JavaScript. Prisma pozwala na efektywne komunikowanie się z bazami danych, eliminując konieczność bezpośredniego korzystania z języka SQL. Framework ten oferuje interfejs programistyczny do definiowania modeli danych, co umożliwia tworzenie zgodnych z typami struktur danych i operacji bazodanowych. Prisma obsługuje różne silniki baz danych, takie jak PostgreSQL, MySQL i SQLite. \autocite{prisma}

Jednym z kluczowych elementów Prisma ORM jest jego zdolność do generowania kodu na podstawie zdefiniowanych modeli danych, co ułatwia utrzymanie spójności pomiędzy kodem a strukturą bazy danych. Ponadto, Prisma wspiera zaawansowane funkcje takie jak relacje między modelami, transakcje bazodanowe oraz zapytania w stylu fluent API.

Dodatkowo, Prisma dostarcza dostarcza narzędzie o nazwie Studio, które umożliwia wizualizację struktury bazy danych oraz wykonywanie zapytań bezpośrednio z poziomu przeglądarki internetowej.
\subsection{Node.js}
Node.js to wieloplatformowe środowisko uruchomieniowe oparte na silniku V8. Pozwala na uruchamianie kodu JavaScript poza przeglądarką internetową. Node.js jest często wykorzystywany do tworzenia serwerów internetowych, jednak może być również używany do tworzenia aplikacji konsolowych. Node.js jest  aktywnie rozwijany, jego obecna wersja to 20.10.0. \autocite{nodejs}
\subsection{PNPM}
PNPM - popularny menedżer pakietów dla Node.js. Jest to narzędzie alternatywne dla podstawowego menedżera pakietów NPM. Pakiety są instalowane globalnie, co pozwala na ich współdzielenie pomiędzy projektami, co przyspiesza proces instalacji i przyczynia się do zajmowania mniejszej ilości miejsca na dysku. \autocite{pnpm}
\subsection{Capacitor}
Capacitor to narzędzie do tworzenia aplikacji mobilnych z wykorzystaniem technologii webowych. Działa on jako most pomiędzy kodem aplikacji internetowej a natywnymi funkcjonalnościami platform mobilnych. Wspiera natywne API dla funkcji takich jak dostęp do aparatu, geolokalizacji, powiadomień czy plików. Zapewnia bezproblemową integrację z popularnymi technologiami to tworzenia aplikacji internetowych takimi jak React, Angular czy Vue. \autocite{capacitor}
\subsection{Visual Studio Code}
Visual Studio Code to darmowy i otwarto-źródłowy  edytor kodu, stworzony przez firmę Microsoft. Aktualna wersja to 1.855.1. Visual Studio Code jest dostępny na systemy Windows, Linux, macOS i przeglądarce. VSCode oferuje integrację z systemami kontroli wersji, co ułatwia skoordynowane i efektywne zarządzanie kodem źródłowym. Oferuje również szeroką gamę rozszerzeń, które pozwalają na dostosowanie edytora do indywidualnych potrzeb. \autocite{vscode}
\subsection{Android Studio}
Android Studio - oficjalne środowisko programistyczne dla platformy Android, stworzone przez Google. Jest to rozbudowane środowisko, które zawiera w sobie wszystkie niezbędne narzędzia do tworzenia aplikacji na platformę Android. Aktualna wersja to 2023.1.1. Narzędzie jest dostępne na systemy Windows, Linux i macOS. \autocite{android}

Program ten zawiera także wbudowane narzędzie do projektowania widoków aplikacji. Środowisko wspiera także emulator systemu Android, który pozwala na testowanie aplikacji na różnych urządzeniach bez konieczności posiadania ich.
\subsection{Git}
Git to rozproszony system kontroli wersji, który został stworzony przez Linusa Torvaldsa w 2005 roku. Cechuje się tym, że jest darmowy, otwarto-źródłowy i bardzo wydajny. Git jest systemem rozproszonym, co oznacza, że każdy programista posiada lokalną kopię repozytorium. Dzięki temu każdy programista może pracować niezależnie od siebie, a zmiany mogą być synchronizowane w późniejszym czasie. \autocite{git}
\subsection{GitHub}
Github to platforma służąca do przechowywania projektów wykorzystujących system kontroli wersji Git. Umożliwia ona współpracę wielu programistów nad jednym projektem. Github oferuje również narzędzia do zarządzania projektem, takie jak tablice kanban, które pozwalają na śledzenie postępu prac. Platforma ta stała się bardzo popularna w kwestii udostępniania kodu źródłowego wielu dużych projektów open-source. \autocite{github}