
Zarządzanie i obsługa wydarzeń od zawsze były wymagającym przedsięwzięciem, stanowiącym wyzwanie dla jednostek oraz instytucji, które pragną dostarczyć efektywne i satysfakcjonujące doświadczenia uczestnikom. Rola wydarzeń w społeczeństwie ewoluowała z prostych zgromadzeń do złożonych i wielowarstwowych struktur, wymagających precyzyjnej koordynacji, ścisłego planowania i efektywnego zarządzania zasobami. W rezultacie organizacja wydarzeń stała się procesem złożonym, który wymaga zaangażowania wielu osób i instytucji, a także wykorzystania specjalistycznych narzędzi i rozwiązań.

Wraz z dynamicznym postępem technologicznym, rozwija się również dziedzina narzędzi wspomagających proces organizacyjny. Nowoczesne aplikacje mobilne oraz platformy internetowe i rozwiązania informatyczne stają się integralną częścią skutecznego zarządzania wydarzeniami. Niemniej jednak obecna oferta na rynku skupia się głównie na potrzebach dużych przedsiębiorstw, które regularnie organizują wydarzenia o znacznej skali, takie jak międzynarodowe koncerty czy masowe festiwale.

Wspomniane rozwiązania, choć wysoce efektywne w specyficznych kontekstach, cechują się również kosztami użytkowania oraz skomplikowaną strukturą, co stawia dodatkowe wyzwania przed mniejszymi organizacjami. W szczególności uczelnie czy organizacje studenckie, które często są inicjatorami wydarzeń o mniejszym zakresie, takich jak jednodniowe konferencje czy warsztaty, napotykają trudności w dostosowaniu ogólnodostępnych narzędzi do swoich specyficznych potrzeb.

Deficyt dostępu do dedykowanych rozwiązań stanowi dla mnie inspirację do stworzenia nowej aplikacji, dostosowanej do specyficznych potrzeb mniejszych organizacji. Mój cel to usprawnienie procesu organizacyjnego oraz podniesienie dostępności i efektywności dla tego typu podmiotów. W tym celu postanowiłem stworzyć nową aplikację, która pozwoli na zarządzanie i obsługę  wydarzeń w sposób prosty i intuicyjny, a także dostosowany do specyficznych potrzeb mniejszych organizacji.